
\subsection{The Partition Coefficient}
\label{sec:sorption}

This analysis investigated the peak dose rate contribution from various 
radionuclides to the partition coefficient of those radionuclides. 

The partition or distribution coefficient, $K_d$, relates the amount of contaminant adsorbed into the 
solid phase of the host medium to the amount of contaminant adsorbed into the 
aqueous phase of the host medium. It is a common empirical coefficient used to 
capture the effects of a number of retardation mechanisms. The coefficient 
$K_d$, in units of $[m^3\cdot kg^{-1}]$, is the ratio of the mass of contaminant in the 
solid to the mass of contaminant in the solution.

The retardation factor, $R_f$, which is the ratio between velocity of water through a 
volume and the velocity of a contaminant through that volume, can be expressed 
in terms of the partition coefficient,

\begin{align}
  R_f &= 1+\frac{\rho_b}{n_e}K_d
  \label{retardation}
  \intertext{where}
  \rho_b &= ~~\mbox{bulk density}[kg\cdot m^{-3}]\nonumber
  \intertext{and}
  n_e &= ~~\mbox{effective porosity of the medium}[\%].\nonumber
\end{align}




\subsubsection{Parametric Range}

The parameters in this model were all set to the default values except a multiplier 
applied to the partitioning $K_d$ coefficients.

The multiplier took the forty values $1\times10^{-9}, 5\times10^{-8}, \cdots 
5\times10^{10}$ Only the far field partition coefficients were altered by this 
factor. Partition coefficients effecting the EDZ and fast pathway were not 
changed.


