
\subsubsection{Calculation of Mass Transfer}
The MixedCell model can operate in four modes, each dictating the method by 
which mass transfer across the inner boundary is calculated.  Those modes 
utilize the source term, Dirichlet, and Neumann interfaces to model prescribed, 
flow, advective flow, or diffusive flow.
Once calculated, that material object is removed from the 
internal component ($m_{ij} = -\dot{m}_i$) such that its internal state can be 
queried accurately in future timesteps.  

For the case in which source term is used on the inner boundary, the mass 
transferred from component $i$ to component $j$ in timestep $t_{n}$ is simply

\begin{align} 
m_{ij} &=  \mathcal{S}_i(t_n) \nonumber
\intertext{where}
\mathcal{S}_i(t_n) &= \mbox{ source term provided by component i at }t_n [kg].
\end{align}

For the case in which Dirichlet is used, the mass transferred is determined by 
advection. One dimensional mass flux due to advection is the speed of flowing 
water, $v_z$, scaled by the concentration of contaminants fixed by the Dirichlet 
boundary condition, $C$, all integrated over the porous, degraded area perpendicular to 
the flow, $\theta dxdy$,

\begin{align}
  m_{ij} &= \int_{t_{n-1}}^{t_n}\int_0^y\int_0^x\theta d v_z \mathcal{D}_i(t_n) dxdydt \label{mixed_adv}
\intertext{which, for the Cyder components, becomes }
m_{ij} &= \theta d v_z C 2rl (t_n - t_{n-1})\nonumber\\
\intertext{where}
\mathcal{D}_i(t_n) &= \mbox{ fixed C from component i at }t_n [kg/m^3]\nonumber\\
r &= \mbox{ radius of the cylinder }[m]\nonumber\\
l &= \mbox{ length of the cylinder }[m].\nonumber
\end{align}

For the case in which Neumann is chosen, the mass transfer is taken to be 
dispersive, 

\begin{align}
  m_{ij} &= \int_{t_{n-1}}^{t_n}\int_0^y\int_0^x -D \theta d \mathcal{N}_i(t_n) dxdydt \label{mixed_adv}\\
         &= \int_{t_{n-1}}^{t_n}\int_0^y\int_0^x -D \theta d \frac{\partial C}{\partial z}\Bigg|_{z=r_j} dxdydt \nonumber
\intertext{which, for the Cyder components, becomes }
m_{ij} &= -D \theta d \frac{\partial C}{\partial z}\Bigg|_{z=r_j} 2rl(t_n - t_{n-1}).\nonumber
\end{align}

\subsubsection{Boundary Interfaces}
The source term of available contaminants is all mass in the available degraded fluid,
\begin{align}
\mathcal{S}_j(t_n) &= m_{df}(t_n). 
\end{align}
The desired boundary conditions can be expressed in terms of $m_{df}$. First, the 
Dirichlet boundary condition is 
\begin{align}
\mathcal{D}_j(t_n) &= C_j(t_n)\nonumber\\ 
 &= \frac{m_{df}(t_n)}{V_{df}(t_n)}.
\label{dirichlet_mixed}
\end{align}

From this boundary condition in combination with global advective velocity 
data, porosity data,  and elemental dispersion coefficient data, all other 
boundary conditions can be found. The Neumann boundary condition generated at 
the external boundary of cell $j$ relies on up to date data from cell $k$ and 
on internal state data from the previous time step, such that 

\begin{align}
\mathcal{N}_j(t_n)&= \frac{dC(t_n)}{dr}\Bigg|_{r=r_j}\nonumber\\ 
                  &= \frac{C_k(r_{k-1/2},t_{n-1}) - C_i(r_{j-1/2}, t_n)}{r_{k-1/2} - r_{j-1/2}}
\label{neumann_mixed}
\intertext{where}
r_{j-1/2} &= r_{j} - \frac{r_{j} - r_{i}}{2}.\nonumber\\
r_{k-1/2} &= r_{k} - \frac{r_{k} - r_{j}}{2}.\nonumber
\end{align}

This expression for the concentration gradient can also be used in the Cauchy 
boundary condition, which relies on the advective velocity and concentration 
profile as well as the concentration gradient,

\begin{align}
v_z C_0 &= \frac{dC(t_n)}{dr}\Big|_{r=r_{j}} + v_{z}C_j(t_n).
\label{cauchy_mixed}
\end{align}
