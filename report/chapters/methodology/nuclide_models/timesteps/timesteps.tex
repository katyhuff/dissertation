\subsection{Implicit Timestepping}\label{sec:timestepping}

4.1.1 timestep algorithm

presume/assume unidirectional flow from inside to outside, therefore:

start with interior flows and work outwards

at each mass transfer interface we desire to solve that flow model with most up-to-date information:

first solve the internal mass distribution model for the interior (source) volume

depending on the mass distribution model of the sink volume

if 0-D model

solve the explicit mass transfer model at the interface

update the mass inventory of the source and sink volumes


else

solve the mass distribution model for the sink volume

determine the implicit mass flux based on the updated sink parameters




Each Component passes some information radially outward to the nested 
Component immediately containing it and some information radially 
inward to the nested Component it contains. 


In the case of radionuclide transport, for example, each Component model
requires information about the radionuclides released from the Component it
immediately contains.  Thus, nuclide release information is passed radially
outward from the waste stream sequentially through each containment layer to
the geosphere. However, the solutions within each Component often rely on the
external boundary conditions of that Component.  Thus, the \Cyder model uses an
implicit timestepping method to arrive at the future state of each Component,
radially outward, as a function of both the past state and the current state. 

That is, in Component $j$, some Component in a nested series, the mass flux 
entering the Component at time $t_n$ is found from the initial state of the cell 
at time $t_n$, the inner boundary 
condition at time $t_n$ and the outer boundary condition at $t_{n-1}$.  

\begin{align}
  \dot{m}_{ij}^n &= f( m_j(t_{n-1}) , BC_i(t_n) , BC_j(t_{n-1}) . . . ) \nonumber\\
  \intertext{where}
  m_{ij}(t_n) &= \mbox{ contaminant mass flux from component i to j }[kg/timestep]\nonumber\\
  BC_i(t_n)  &= \mbox{ inner conditions at }r_i\mbox{, and time }t_n \nonumber \\
  BC_j(t_{n-1})  &= \mbox{ outer conditions at }r_j\mbox{, and time }t_{n-1} \nonumber\\
  f &= \mbox{ functional form of contaminant transport into j. }\nonumber
\end{align}

Once the mass flux into the component is found, the mass is removed from the 
inner cell, updating its state in preparation for the next time step.

\begin{align}
  m_i^\dagger(t_n)  &= m_i(t_n)  - m_{ij}(t_n) 
  \intertext{where}
  m_i^\dagger(t_n)  &= \mbox{ updated mass in component i }[kg]
\end{align}

In this way, the contained mass in the component is described as
\begin{align}
  m_j(t_n)  &= m_j(t_{n-1})  + \dot{m}_j(t_n) . \nonumber
\end{align}

Resulting concentration profiles across the component can then be calculated 
and one can solve, numerically, for the outer boundary condition at $t_n$ 

\begin{align}
  BC_j(t_n) &= g\left( m_j(t_n) , C_j(t_n) \right)\nonumber\\
  g &= \mbox{functional form of contaminant transport across j}\nonumber
\end{align}

This boundary condition can, in turn, be used by the component external to it, $k$ as the $t_n$ 
inner boundary condition of its own solution and so on.

