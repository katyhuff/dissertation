\chapter{Introduction}\label{ch:introduction}

% The scope of the work includes software and analysis.

The scope of this work includes development and validation of \Cyder, a generic 
repository software library for modeling various long-term disposal system concepts 
for nuclear material. \Cyder is integrated with the \Cyclus 
computational fuel cycle systems analysis platform in order to inform 
repository performance metrics with respect to candidate fuel cycle options.  
By abstraction of more detailed models, this work captures the dominant physics 
of radionuclide and heat transport phenomena affecting repository performance 
in various geologic media and as a function of arbitrary spent fuel 
composition. 

\section{Motivation} 
% Waste is a problem
% Decisionmakers are contemplating many fuel cycle options
% Decisionmakers are contemplating many repository options
% Interfacing between FCO/SA campaign and UFD campaign

\begin{frame}[ctb!]
  \frametitle{Future Disposal System Options}
   \begin{minipage}{0.44\textwidth}
     \begin{figure}[h!]
         \includegraphics[width=0.8\textwidth]{./images/saltNewScientist.eps}
         \caption{U.S. Salt Deposits, ref. \cite{newscientist_where_2011}.}
     \end{figure}
     \begin{figure}[h!]
         \includegraphics[width=0.8\textwidth]{./images/clayGonzales.eps}
         \caption{U.S. Clay Deposits, ref. \cite{gonzales_shales_1985}.}
     \end{figure}
   \end{minipage}
   \hspace{0.01cm}
   \begin{minipage}{0.44\textwidth}
     \begin{figure}[h!]
         \includegraphics[width=0.8\textwidth]{./images/boreholeNewScientist.eps}
         \caption{U.S. Crystalline Basement, ref.  \cite{newscientist_where_2011}.}
     \end{figure}
     \begin{figure}[h!]
         \includegraphics[width=0.8\textwidth]{./images/graniteBush.eps}
         \caption{U.S. Granite Beds, ref. \cite{bush_economic_1976}.}
     \end{figure}
   \end{minipage}
\end{frame}


\begin{frame}[ctb!]
  \frametitle{Future Fuel Cycle Options}
    \input{fco_tab}
\end{frame}


\begin{frame}[ctb!]
\frametitle{Cyder Contributions}

This work has provided a platform capable of bridging the gap between fuel cycle 
simulation and repository performance analysis.

  \begin{itemize}
  \item \Cyder acheived integration with a fuel cycle simulator.
  \item Conducted thermal transport sensitivity analyses. \cite{huff_numerical_2012, huff_benchmarking_2012}
  \item Conducted contaminan transport sensitivity analyses. \cite{huff_key_2012}
  \item Abstracted physical models of thermal and contaminant transport. \cite{huff_hydrologic_2013}
  \item Demonstrated dominant physics of those models in \Cyder, integrated 
  with \Cyclus. \cite{huff_dyanmic_2013, huff_cyclus_2013}
  \item Published source code, documentation, and testing to facilitate 
  extension by external developers. \cite{huff_cyder_2013}
  \end{itemize}
\end{frame}

\end{frame}


\section{Methodology} 
% Overview : 
% put a repository model into cyclus
% that is capable of distinguishing between disposal choices
% and fuel cycle choices
% but is still speedy. 

\begin{frame}[ctb!]
  \frametitle{Methodology : Modularity }
  Interchangeable Subcomponents, integration with fuel cycle model.
\end{frame}

\begin{frame}[ctb!]
  \frametitle{Methodology : Abstraction for Efficiency}
  Abstratction
  GPAM
  GDSM
\end{frame}

\section{Outline}
% Summarize document

A literature review, Chapter \ref{ch:litrev}, presents background material that organizes and 
reports upon previous relevant work. First it summarizes the state of the art of 
repository modeling integration within current systems analysis tools. It then 
describes current domestic and international disposal system option space. 
Next, the literature review focuses upon current analytical and 
computational modeling of radionuclide and heat transport through various waste 
forms, engineered barrier systems, and geologies of interest.  It also 
addresses previous efforts in generic geologic environment repository modeling in order to 
categorize and characterize detailed computational models of radionuclide and 
heat transport available for abstraction and validaton efforts.

% modeling paradigm

Chapter \ref{ch:paradigm} details the computational paradigm of the \Cyclus 
systems analysis platform and \Cyder repository model which constitute this work. 
It describes the \Cyclus fuel cycle simulation context which drives fundamental 
\Cyder design descisions as well as the modular paradigm emphasized in both 
\Cyclus and \Cyder. 

% methodology

Chapter \ref{ch:methodology} describes the radionuclide transport and heat 
transport models that were implemented. Models representing engineered barrier 
and geological disposal system components are defined by their interfaces and 
their relationships as interconnected modules, distinctly defined, but coupled. 
This modular paradigm allows exchange  of technological options (i.e. 
borosilicate glass and concrete waste forms) for comparison but also exchange of 
models for the same technological option with varying levels of detail.  

% Test Cases and Benchmarks

Chapter \ref{ch:demonstration} describes simulation cases conducted to 
demonstrate radionuclide transport and thermal capacity analysis capabilities in \Cyder. It 
also describes verification and validation procedures which benchmarked \Cyder 
behavior against analytic solutions and more detailed models.


% Conclusions
Finally, in Chapter \ref{ch:conclusions}, contributions to the field and 
suggested future work are summarized. 



%%%%%%%% %%%%%%%% %%%%%%%% %%%%%%%% %%%%%%%% %%%%%%%% %%%%%%%% %%%%%%%%
%%%%%%%% %%%%%%%% %%%% These chapters may be saved for the thesis . . .  
%%%%%%%% %%%%%%%% %%%%%%%% %%%%%%%% %%%%%%%% %%%%%%%% %%%%%%%% %%%%%%%% 

% Chapter \ref{ch:ebs} will adapt existing models and data to the development of 
% concise dominant physics waste form, waste package, and other engineered 
% barriers (i.e., bentonite or cementitious materials) models appropriate for 
% treatment of key radionuclides within the waste streams.  Material/barrier 
% degradation, radionuclide release, and radionuclide transport, and thermal 
% processes and effects will be included, as necessary, in the concise 
% representations that will be developed for subsequent use in the system-level 
% architecture. A range of waste forms, waste package materials, and other 
% engineered barrier materials (buffer, backfill) under consideration by the 
% DOE-NE FCT program (SWF and UFD campaigns) will be evaluated. The concise 
% dominant physics models will include appropriate load limiting factors of the 
% stabilizing medium and waste packaging including such as waste composition, 
% chemical form, and heat generation.
% 
% Chapter \ref{ch:extension} will discuss the future work necessary to extend 
% developed models to comprehensively cover the potential disposal system option 
% space. The path forward for extension of the geological base case model to 
% cover all five geologic concepts of interest (clay/shale, granite, salt, and 
% deep boreholes) will be discussed. Similarly, gaps in waste form and 
% engineered barrier system models and data will be addressed and a plan for 
% data and model coverage for that options space will be described.

