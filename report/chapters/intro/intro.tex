\chapter{Introduction}\label{ch:introduction}

% The scope of the work includes software and analysis.

The scope of this work includes implementation of Cyder, a software library of medium 
fidelity models to comprehensively represent various long-term disposal system 
concepts for nuclear material. This software library is integrated with the 
\Cyclus computational fuel cycle systems analysis platform in order to inform repository 
performance metrics with respect to candidate fuel cycle options. By abstraction 
of more detailed models, this work captures the dominant physics of 
radionuclide and heat transport phenomena affecting repository performance in 
various geologic media and as a function of arbitrary spent fuel composition. 

\section{Motivation} 
% Waste is a problem
% Decisionmakers are contemplating many fuel cycle options
% Decisionmakers are contemplating many repository options
% Interfacing between FCO/SA campaign and UFD campaign

\begin{frame}[ctb!]
  \frametitle{Future Disposal System Options}
   \begin{minipage}{0.44\textwidth}
     \begin{figure}[h!]
         \includegraphics[width=0.8\textwidth]{./images/saltNewScientist.eps}
         \caption{U.S. Salt Deposits, ref. \cite{newscientist_where_2011}.}
     \end{figure}
     \begin{figure}[h!]
         \includegraphics[width=0.8\textwidth]{./images/clayGonzales.eps}
         \caption{U.S. Clay Deposits, ref. \cite{gonzales_shales_1985}.}
     \end{figure}
   \end{minipage}
   \hspace{0.01cm}
   \begin{minipage}{0.44\textwidth}
     \begin{figure}[h!]
         \includegraphics[width=0.8\textwidth]{./images/boreholeNewScientist.eps}
         \caption{U.S. Crystalline Basement, ref.  \cite{newscientist_where_2011}.}
     \end{figure}
     \begin{figure}[h!]
         \includegraphics[width=0.8\textwidth]{./images/graniteBush.eps}
         \caption{U.S. Granite Beds, ref. \cite{bush_economic_1976}.}
     \end{figure}
   \end{minipage}
\end{frame}


\begin{frame}[ctb!]
  \frametitle{Future Fuel Cycle Options}
    \input{fco_tab}
\end{frame}


\begin{frame}[ctb!]
\frametitle{Cyder Contributions}

This work has provided a platform capable of bridging the gap between fuel cycle 
simulation and repository performance analysis.

  \begin{itemize}
  \item \Cyder acheived integration with a fuel cycle simulator.
  \item Conducted thermal transport sensitivity analyses. \cite{huff_numerical_2012, huff_benchmarking_2012}
  \item Conducted contaminan transport sensitivity analyses. \cite{huff_key_2012}
  \item Abstracted physical models of thermal and contaminant transport. \cite{huff_hydrologic_2013}
  \item Demonstrated dominant physics of those models in \Cyder, integrated 
  with \Cyclus. \cite{huff_dyanmic_2013, huff_cyclus_2013}
  \item Published source code, documentation, and testing to facilitate 
  extension by external developers. \cite{huff_cyder_2013}
  \end{itemize}
\end{frame}

\end{frame}


\section{Methodology} 

% Scope

In this work, concise dominant physics models suitable for system level fuel 
cycle codes were developed by comparison of analytical models with more 
detailed repository modeling efforts. The ultimate result of this work is a 
software library capable of assessing a wide range of combinations 
of fuel cycle alternatives, potential waste forms, repository design concepts, 
and geological media. 

% Selection of base set of subcomponents to model

Current candidate repository concepts have been investigated and reviewed here 
in order to arrive at a fundamental set of components to model. A preliminary 
set of combinations of fuel cycles, repository concepts, and geological 
environments has 
been chosen that fundamentally captures the domestic option space. 
Specifically, three candidate geologies and four corresponding repository 
concepts under consideration by the \gls{DOE} \gls{UFD} campaign have been 
chosen for modeling in this work. These have been implemented to interface
with \Cyclus simulations of a canonical set of potential fuel cycles within 
three broad candidate scenarios put forth by the \gls{US} \gls{DOE}.

% Review of available analytical models

A review and characterization of the physical mechanisms by which radionuclide 
and thermal transport take place within the materials and media under 
consideration was first undertaken. Potential analytical models to represent  
these phenomena were investigated and categorized within the literature review. 

% Review of available current detailed tools

A review and characterization of current detailed computational tools for
repository focused analysis was next conducted. Both international and domestic
repository modeling efforts were summarized within the literature review. Of 
these, candidate computational tools with which to perform abstraction and 
regression analyses were identified. Specifically, a suite of \gls{GDSM} tools 
under development by the \gls{DOE} \gls{UFD} campaign has informed radionuclide 
transport models and a pair of corresponding thermal analysis codes has 
informed the thermal models.
 
% For each component, abstraction of current tools and analytical models

This suite of subcomponent modules, appropriate for use within the \Cyclus fuel 
cycle simulator, has included the development of a robust architecture 
within the repository module that allows for interchangeable loading of 
system components.  Within the system components, dominant physics is 
modeled based on domain appropriate approximation of analytical models and 
supported by abstraction with the chosen \gls{GDSM} tools and thermal tools. 

% Abstraction = Analytic Solutions + Regression Approximation

These concise models are a combination of two components: 
semi-analytic mathematical models that represent a simplified description of 
the most important physical phenomena, and semi-empirical models that reproduce 
the results of detailed models.  By combining the complexity of the analytic 
models and regression against numerical experiments, variations were limited 
between two models for the same system.  Different approaches have been been 
compared in this work, with final modeling choices balancing the accuracy and 
efficiency of the possible implementations.  

% For example, Geology.

Specifically, these models focus on the hydrology and thermal 
physics that dominate radionuclide transport and heat response in candidate 
geologies as a function of radionuclide release and heat generation over long 
time scales. Dominant transport mechanism (advection or 
diffusion) and disposal site water chemistry (redox state) provide primary 
differentiation between the different geologic media under consideration. In 
addition, the concise models are capable of roughly adjusting release 
pathways according to the characteristics of the natural system (both the host 
geologic setting and the site in general) and the engineered system (such as package 
loading arrangements, tunnel spacing, and engineered barriers).

% Abstraction in repository environment

The abstraction process in the development of a geological environment model 
employs the comparison of semi-analytic thermal and hydrologic models and 
analytic regression of rich code results from more detailed models as well as 
existing empirical geologic data. Such results and data will be derived 
primarily from the \gls{UFD} campaign \glspl{GDSM} and data, as well as European 
efforts such as the RED-IMPACT assessment and \gls{ANDRA} Dossier efforts 
\cite{von_lensa_red-impact_2008, andra_argile:_2005, clayton_generic_2011} . 

% Example, heat from WPs along repo drifts.

%For example, analytic models and semi-empirical models are available (i.e.  
%specific temperature integrals in ref. \cite{li_methodology_2006} and the 
%specific temperature change index in ref. \cite{radel_effect_2007}) that 
%approximate thermal response from heat generation in waste packages as linear
%along repository drifts. These analyses arrive at a thermal evolution over time 
%at any location in the rock by superpositioning and integration. Subsequently, 
%their results can be converted easily into conventional line loading and areal 
%power density metrics.  Such analytic models will first be assessed to determine
%likely parameters upon which thermal response will rely (e.g. tunnel spacing, 
%radionuclide inventories, etc.).
%
% Regression.

Thereafter, a regression analysis concerning those parameters has been undertaken 
with available detailed models (e.g. 2D and 3D finite element thermal 
performance assessment codes) to further characterize the 
parametric dependence of thermal loading in a specific geologic environment.  

% Incorporation of empirical data

Finally, the thermal behavior of a repository model so developed depends on 
empirical data (e.g.  heat transfer coefficients, hydraulic conductivity).  
Representative values have been made available within the dominant 
physics models and rely on existing empirical data concerning the specific 
geologic environment being modeled (i.e. salt, clay/shale, and granite). 

% Abstraction for WFs, WPs, EBS, etc. will be the same

A similar process was followed for radionuclide transport models.  The 
abstraction process in the development of waste form, package, and engineered 
barrier system models will be analogous to the abstraction process of repository 
environment models. Concise models have resulted from employing the comparison of 
semi-analytic models of those systems with regression analysis of rich code in 
combination with existing empirical material data.

% Example, radionuclide release

%For instance, in the case of radionuclide release from waste packages, analytic 
%models of radionuclide release (e.g.  congruent or solubility limited) will 
%first be assessed to determine likely parameters upon which radionuclide release 
%will rely (e.g.  radionuclide concentration, water flow rates, etc.) 
%\cite{kawasaki_congruent_2004}.  Again, regression analysis concerning those 
%parameters will be undertaken with available detailed models to further 
%characterize the parametric dependence of radionuclide release from specific 
%waste packages.  Finally, the radionuclide release model so developed will 
%depend on empirical data (e.g. the waste form dissolution rate).  Determination 
%of representative values to make available within the dominant physics model 
%will rely on existing empirical data concerning the specific waste form  
%materials being modeled (i.e. long time scale extrapolations of known glass 
%degradation rates).  

% Coupling is confusing

Coupling effects between components were considered carefully.  In 
particular, given the important role of temperature in the system, thermal 
coupling between the models for the engineered system and the geologic system 
were considered. Thermal dependence of radionuclide release and transport as 
well as package degradation were analyzed to determine the 
magnitude of coupling effects in the system.

% Iterate

The full abstraction process was be iterated to achieve a balance between 
calculation speed and simulation detail. Model improvements during this stage 
sought a level of detail appropriate for informative comparison of subcomponents, but 
with sufficient speed to enable systems analysis. 

% Test using canonical data for those subcomponents

By varying input parameters and comparing with corresponding results from 
detailed tools, each model's behavior on its full parameter domain was 
validated.


\section{Outline}

% Summarize document

A literature review, Chapter \ref{ch:litrev}, presents background material that organizes and 
reports upon previous relevant work. First it summarizes the state of the art of 
repository modeling integration within current systems analysis tools. It then 
describes current domestic and international disposal system concepts and
geologies.  Next, the literature review focuses upon current analytical and 
computational modeling of radionuclide and heat transport through various waste 
forms, engineered barrier systems, and geologies of interest.  It also 
addresses previous efforts in generic geologic environment repository modeling in order to 
categorize and characterize detailed computational models of radionuclide and 
heat transport available for regression analysis.

% modeling paradigm

Chapter \ref{ch:paradigm} details the computational paradigm of the \Cyclus 
systems analysis platform and Cyder repository model which constitute this work. 
It describes the \Cyclus fuel cycle simulation context which drives fundamental 
Cyder design descisions. 

% methodology

Chapter \ref{ch:methodology} describes the radionuclide and heat transport 
models that were implemented. Models representing waste form, waste package, 
buffer, backfill, and engineered barrier componeents are defined by their 
interfaces and their relationships as interconnected modules, distinctly 
defined, but coupled. This modular paradigm allows exchange  of technological 
options (i.e. borosilicate glass and concrete waste forms) for comparison but 
also exchange of models for the same technological option with varying levels 
of detail.  

% Test Cases and Benchmarks

Chapter \ref{ch:demonstration} describes demonstration cases conducted to 
demonstrate radionuclide transport and thermal capacity analysis in Cyder. It 
also describes verification and validation procedures which benchmarked Cyder 
behavior against more detailed models.


Finally, in Chapter \ref{ch:conclusions}, contributions to the field and 
suggested future work are summarized. 

% Conclusions

Chapter \ref{ch:conclusions} will 

%%%%%%%% %%%%%%%% %%%%%%%% %%%%%%%% %%%%%%%% %%%%%%%% %%%%%%%% %%%%%%%%
%%%%%%%% %%%%%%%% %%%% These chapters may be saved for the thesis . . .  
%%%%%%%% %%%%%%%% %%%%%%%% %%%%%%%% %%%%%%%% %%%%%%%% %%%%%%%% %%%%%%%% 

% Chapter \ref{ch:ebs} will adapt existing models and data to the development of 
% concise dominant physics waste form, waste package, and other engineered 
% barriers (i.e., bentonite or cementitious materials) models appropriate for 
% treatment of key radionuclides within the waste streams.  Material/barrier 
% degradation, radionuclide release, and radionuclide transport, and thermal 
% processes and effects will be included, as necessary, in the concise 
% representations that will be developed for subsequent use in the system-level 
% architecture. A range of waste forms, waste package materials, and other 
% engineered barrier materials (buffer, backfill) under consideration by the 
% DOE-NE FCT program (SWF and UFD campaigns) will be evaluated. The concise 
% dominant physics models will include appropriate load limiting factors of the 
% stabilizing medium and waste packaging including such as waste composition, 
% chemical form, and heat generation.
% 
% Chapter \ref{ch:extension} will discuss the future work necessary to extend 
% developed models to comprehensively cover the potential disposal system option 
% space. The path forward for extension of the geological base case model to 
% cover all five geologic concepts of interest (clay/shale, granite, salt, and 
% deep boreholes) will be discussed. Similarly, gaps in waste form and 
% engineered barrier system models and data will be addressed and a plan for 
% data and model coverage for that options space will be described.

