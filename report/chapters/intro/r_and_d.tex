
The DOE-NE Fuel Cycle Technology (FCT) program has three groups of relevance to 
this effort: these are the \gls{UFD}, the \gls{SWF}, and \gls{FCO} (previously 
Systems Analysis) campaigns.  
The \gls{UFD} campaign is conducting the \gls{RDD} related to the storage, 
transportation, and disposal of radioactive wastes generated under both the 
current and potential advanced fuel cycles.  The SWF campaign is conducting 
\gls{RDD} on potential waste forms that could be used to effectively isolate 
the 
wastes that would be generated in advanced fuel cycles.  The \gls{SWF} and
\gls{UFD} campaigns are developing the fundamental tools and information base 
regarding the performance of waste forms and geologic disposal systems.  The 
\gls{FCO} campaign is developing the overall fuel cycle simulation tools and 
interfaces with the other FCT campaigns, including \gls{UFD}.  

This effort has interfaced with those campaigns to develop the higher level
dominant physics representations for use in fuel cycle system analysis tools.
Specifically, this work has leveraged conceptual framework development and
primary data collection underway within the Used Fuel Disposition Campaign as
well as work by Radel, Wilson, Bauer et. al. to model repository behavior as a
function of the contents of the waste \cite{radel_effect_2007}.  It then 
incorporated dominant physics process models into the \Cyclus computational 
fuel cycle analysis tool \cite{wilson_cyclus:_2012} based on sensitivity analysis 
utilizing those detailed tools.
