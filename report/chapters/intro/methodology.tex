% Scope

In this work, concise dominant physics thermal and radionuclide transport models 
were developed by comparison of analytical models with more 
detailed repository modeling efforts. The ultimate result of this work is a 
software library capable of assessing a wide range of combinations 
of fuel cycle alternatives, potential waste forms, repository design concepts, 
and geological media. Within this software library a suite of basic 
capabilities have been demonstrated and validated and a few advanced 
features are operational. 

% Selection of base set of subcomponents to model

Current candidate repository concepts have been investigated and reviewed here 
in order to arrive at a fundamental set of components to model. A preliminary 
set of combinations of fuel cycles, repository concepts, and geological 
environments has 
been chosen that fundamentally captures the domestic option space. 
Specifically, three candidate geologies and four corresponding repository 
concepts under consideration by the \gls{DOE} \gls{UFD} campaign have been 
chosen for modeling in this work. These have been implemented to interface
with \Cyclus simulations of a canonical set of potential fuel cycles within 
three broad candidate scenarios put forth by the \gls{US} \gls{DOE}.

% Review of available analytical models

A review and characterization of the physical mechanisms by which radionuclide 
and thermal transport take place within the materials and media under 
consideration was also undertaken. Potential analytical models to represent  
these phenomena were investigated and categorized within the literature review. 

% Review of available current detailed tools

An assessment of current detailed computational tools for
repository focused analysis was next conducted. Both international and domestic
repository modeling efforts were summarized within the literature review. Of 
these, candidate computational tools with which to perform abstraction and 
regression analyses were identified. Specifically, a suite of \gls{GDSM} tools 
under development by the \gls{DOE} \gls{UFD} campaign has informed radionuclide 
transport models and a pair of corresponding thermal analysis codes has 
informed the thermal models.
 
% For each component, abstraction of current tools and analytical models

This suite of subcomponent modules, appropriate for use within the \Cyclus fuel 
cycle simulator, has included the development of a robust architecture 
within the repository module that allows for interchangeable loading of 
system components.  Within the system components, dominant physics is 
modeled based on domain appropriate approximation of analytical models and 
supported by abstraction with the chosen \gls{GDSM} tools and thermal tools. 

% Abstraction = Analytic Solutions + Regression Approximation

These concise models are a combination of two components: 
semi-analytic mathematical models that represent a simplified description of 
the most important physical phenomena, and semi-empirical models that reproduce 
the results of detailed models.  By combining the complexity of the analytic 
models and regression against numerical experiments, variations were limited 
between two models for the same system.  Different approaches have been been 
compared in this work, with final modeling choices balancing the accuracy and 
efficiency of the possible implementations.  

% For example, Geology.

Specifically, these models focus on the hydrology and thermal 
physics that dominate radionuclide transport and heat response in candidate 
geologies as a function of radionuclide release and heat generation over long 
time scales. Dominant transport mechanism (advection or 
diffusion) and disposal site water chemistry (redox state) provide primary 
differentiation between the different geologic media under consideration. In 
addition, the concise models are capable of roughly adjusting release 
pathways according to the characteristics of the natural system (both the host 
geologic setting and the site in general) and the engineered system (such as package 
loading arrangements, tunnel spacing, and engineered barriers).

% Abstraction in repository environment

The abstraction process in the development of engineered barrier and  geological 
environment models employs the comparison of semi-analytic thermal and 
hydrologic models and analytic regression of rich code results from more 
detailed models as well as existing empirical geologic and material data data. 
Such results and data are derived primarily from the \gls{UFD} campaign 
\glspl{GDSM} and data, as well as European efforts such as the RED-IMPACT 
assessment and \gls{ANDRA} Dossier efforts \cite{von_lensa_red-impact_2008, 
andra_argile:_2005, clayton_generic_2011} . 

% Abstraction for WFs, WPs, EBS, etc. will be the same

% Example, heat from WPs along repo drifts.

%For example, analytic models and semi-empirical models are available (i.e.  
%specific temperature integrals in ref. \cite{li_methodology_2006} and the 
%specific temperature change index in ref. \cite{radel_effect_2007}) that 
%approximate thermal response from heat generation in waste packages as linear
%along repository drifts. These analyses arrive at a thermal evolution over time 
%at any location in the rock by superpositioning and integration. Subsequently, 
%their results can be converted easily into conventional line loading and areal 
%power density metrics.  Such analytic models will first be assessed to determine
%likely parameters upon which thermal response will rely (e.g. tunnel spacing, 
%radionuclide inventories, etc.).
%
% Regression.

Thereafter, a regression analysis concerning those parameters has been 
undertaken with available detailed models ( 2D finite element thermal 
performance assessment code \cite{huff_thermal_2012} and a computational 
analytical model \cite{greenberg_application_2011}) to further characterize the 
parametric dependence of thermal loading in a specific geologic environment.  

% Incorporation of empirical data

Finally, the thermal behavior of the \Cyder model depends on empirical data 
(e.g. thermal conductivity, thermal diffusivity, material density, etc.). 
Accordingly, thermal transport in \Cyder relies on existing empirical data 
concerning the specific geologic environment being modeled (i.e. salt, 
clay/shale, and granite) for which representative values have been made 
available.


%For instance, in the case of radionuclide release from waste packages, analytic 
%models of radionuclide release (e.g.  congruent or solubility limited) will 
%first be assessed to determine likely parameters upon which radionuclide release 
%will rely (e.g.  radionuclide concentration, water flow rates, etc.) 
%\cite{kawasaki_congruent_2004}.  Again, regression analysis concerning those 
%parameters will be undertaken with available detailed models to further 
%characterize the parametric dependence of radionuclide release from specific 
%waste packages.  Finally, the radionuclide release model so developed will 
%depend on empirical data (e.g. the waste form dissolution rate).  Determination 
%of representative values to make available within the dominant physics model 
%will rely on existing empirical data concerning the specific waste form  
%materials being modeled (i.e. long time scale extrapolations of known glass 
%degradation rates).  

% Iterate
% Test using canonical data for those subcomponents 

The full abstraction process was iterated to achieve a balance between 
calculation speed and simulation detail. Model improvements during this stage 
sought a level of detail appropriate for informative comparison of subcomponents, but 
with sufficient speed to enable systems analysis. 
By varying input parameters and comparing with corresponding results from 
detailed tools, the behavior of each model on its parametric domain was 
validated.
