
% DOE is interested in many fuel cycle options. 
% DOE is considering 3 main options, each of which pose different waste 
% management challenges . . .

As the United States and other nations seek to develop technologies and strategies to support a 
sustainable future for nuclear energy, various fuel cycle strategies and 
corresponding disposal system options are being considered.  Specifically, the 
domestic fuel cycle option space under current consideration is described in 
terms of three distinct fuel cycle categories with the monikers Once Through, 
Full Recycle, and Modified Open. Each category presents unique disposal system 
design challenges. Systems analyses for evaluating these options must 
be undertaken in order to inform a national decision to deploy a comprehensive 
fuel cycle system by 2050 \cite{doe_nuclear_2010}. 

% Once through presents capacity issues . . . 

The Once-Through Cycle category includes fuel cycles similar to the fuel cycle 
currently deployed in the United States, utilizing light water reactors and 
direct disposal of spent nuclear fuel in a geologic repository.
Such fuel cycles neglect reprocessing and present challenges associated with 
high volumes of minimally treated spent fuel streams.  In in a geologic 
repository a business as usual 
scenario, conventional power reactors comprise the majority of nuclear energy 
production. 
Calculations from the Electric Power Research Institute corroborated by 
the \gls{US} \gls{DOE} in 2008 indicate that without an increase in the statutory 
capacity limit of the \gls{YMR}, continuation of the current Once Through fuel 
cycle will generate a volume of spent fuel that will necessitate
the siting of an additional federal geologic repository to accommodate spent 
fuel \cite{kessler_room_2006, doe_report_2008}. 

% Full Recycle presents the issue of variable waste streams. . .

A Full Recycle option, on the other hand, requires the research, development, 
and deployment of partitioning, transmutation, and advanced reactor technology 
for the reprocessing of used nuclear fuel.  In this scheme, conventional 
once-through reactors will be phased out in favor of advanced transmutation 
technologies. All fuel in the Full Recycle strategy will be 
reprocessed using an accelerator driven system or by 
cycling through an advanced fast reactor. Such fuel may undergo partitioning, 
the losses from which will require waste treatment and ultimate disposal in a 
repository. Thus, a repository under the Full Recycle scenario must support
a waste stream composition that is highly variable during transition periods as 
well as myriad waste forms and packaging associated with isolation of differing 
waste streams.

% Modified Open presents both problems. . . 

Finally, the Modified Open Cycle category of options includes a variety of fuel 
cycle options that fall between once through and fully closed. Advanced fuel 
cycles such as deep burn and small modular reactors will be considered as will 
partial recycle options.  Partitioning and reprocessing strategies, however, 
will be limited to simplified chemical separations and volatilization. This 
scheme presents a dual challenge in which spent fuel volumes 
and composition will both vary dramatically among various possibilities within 
this scheme \cite{doe_nuclear_2010} .

% Various waste streams require various WFs and WPs 

Clearly, the waste streams resulting from potential fuel cycles present an 
array of corresponding waste disposition, packaging, and engineered barrier 
system options. Differing spent fuel composition, partitioning, transmutation, 
and chemical processing decisions upstream in the fuel cycle may demand differing 
natural and engineered barrier requirements during disposal. The 
capability to model thermal and radionuclide transport phenomena of 
arbitrary isotopic compositions is therefore required. This work has produced a 
disposal system simulator that meets this need. 

