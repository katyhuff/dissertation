\subsection{Thermal Conductivity Sensitivity Validation}\label{sec:conductivity}
The thermal conductivity, $K_{th}$ of geologic repository host media impacts 
the speed of transport, and therefore the time evolution of thermal energy 
deposition, in the medium. 

\FloatBarrier
\subsubsection{LLNL Model Results}
In the creation of the \gls{STC} database, the thermal conductivity was varied 
across a broad domain for each isotope, $i$, package spacing, $s$, limiting 
radius $r_{calc}$, and thermal diffusivity $\alpha_{th}$, considered.  By 
varying the thermal conductivity of the repository model from 0.1 to 4.5
$[W\cdot m^{-1} \cdot K^{-1}]$, this sensitivity analysis succeeds in capturing the domain of 
thermal conductivities witnessed in high thermal conductivity salt deposits as 
well as low thermal conductivity clays.

  \begin{figure}[htbp!]
    \begin{center}
      \includegraphics[width=0.7\textwidth]{./chapters/demonstration/conductivity/conductivity.eps}
    \end{center}
    \caption[$K_{th}$ Sensitivity in LLNL Model]{Increased thermal conductivity 
      decreases the temperature (here represented by STC) at the limiting 
    radius.}
    \label{fig:Cm242Kth_alpha_low}
  \end{figure}

Figure \ref{fig:Cm242Kth_alpha_low} shows the trend in which increased thermal 
conductivity of a medium decreases temperature change 
in the near field. This indicates, then that thermal conductivity is 
an important parameter for repository geologic medium selection. 

\FloatBarrier
\subsubsection{Cyder Results}
In a similar analysis, the thermal conductivity was investigated. Figure 
\ref{fig:conductivity_cyder} shows that the same trend noted for the LLNL model 
was noted in the 
\Cyder model.

  \begin{figure}[htbp!]
    \begin{center}
      \includegraphics[width=0.7\textwidth]{./chapters/demonstration/conductivity/conductivity_cyder.eps}
    \end{center}
    \caption[$K_{th}$ Sensitivity in Cyder]
    {Cyder results agree with those of the LLNL model. Increased $K_{th}$ 
    decreases 
    temperature change at the limiting radius. The above example thermal 
    profile results from 10kg of $^{242}Cm$, $\alpha_{th}=2\times 10^{-7}$, 
    $s=5m$, 
    and $r_{lim}=0.25m$.}
    \label{fig:conductivity_cyder}
  \end{figure}


In additional dual parameter studies, the importance of the thermal conductivity 
was compared both with the spacing between waste packages and the limiting 
radius. Figures \ref{fig:kr} and \ref{fig:ks} validate the trend noted above 
that increased thermal conductivity of a medium decreases temperature change in 
the near field.  Additionally, analysis with the \Cyder STC database 
demonstrates the way in which the importance of spacing and the importance of 
the limiting radius decrease with increasing $K_{th}$.

\begin{figure}[htbp!]
\begin{center}
\includegraphics[width=0.7\textwidth]{./chapters/demonstration/conductivity/kr.eps}
\end{center}
\caption[$K_{th}$ vs. $r_{lim}$ Sensitivity in Cyder]{\Cyder results agree with 
those of the LLNL model. The importance of the limiting radius decreases with 
increased $K_{th}$. The above example thermal profile results from 10kg of 
$^{242}Cm$}
\label{fig:kr}
\end{figure}

\begin{figure}[htbp!]
\begin{center}
\includegraphics[width=0.7\textwidth]{./chapters/demonstration/conductivity/ks.eps}
\end{center}
\caption[$K_{th}$ vs. Waste Package Spacing Sensitivity in Cyder]{Cyder results agree with 
those of the LLNL model. The importance of the limiting radius decreases with 
increased $K_{th}$. The above example thermal profile results from 10kg of 
$^{242}Cm$}
\label{fig:ks}
\end{figure}

