Each engineered barrier component within the Generic Repository calculates nuclide 
transport using a model selected from those presented in this chapter. In order 
to be interchangeable within the simulation, these components must have 
identical nuclide transport interfaces. 

Nuclide models may rely on a number of boundary condition forms. 
Each nuclide model interface must therefore provide sufficient boundary 
condition information to support the calculation methods of all other nuclide 
models. That is, each must provide a superset of the boundary condition forms 
used in other models. These include Dirichlet, specified species concentration along 
the boundary, Neumann, concentration gradients along the boundary, 
and Cauchy, a combination of the two that provides a concentration flux along the 
boundary.

For all nuclide models, mass must be conserved. Thus, all nuclide models share a 
mass balance paradigm. 

First, since time is discrete within \Cyclus, the remaining mass in a cell at the 
end of timestep $t_n$ is the mass of the cell at the beginning of timestep 
$t_{n+1}$.

For all components ($k$) the mass balance is simply the sum over time of 
incoming and outgoing mass. No component has more than one external component 
($l$), but some have many internal components ($j$, children). The mass balance 
equation for cell $k$ is then

\begin{align}
m_k^{n+1} &= m_k^0 + \sum_{i=0}^n\left[ \sum_{j=0}^{children}\left( 
m_{j,k}^i - m_{k,l}^i\right)(t_n-t_{n-1})\right]
\label{mass_balance}
\end{align}

For a Dirichlet boundary condition, each nulcide transport model must represent 
a species concentration along the boundary of the representative volume, 

\begin{align}
  C(x,y,z,t) &= C_0(x,y,z,t)\hspace{1mm}\mbox{ for } (x,y,z) \in \Gamma
  \intertext{where}
  C(x,y,z,t) &= \mbox{ concentration at the boundary } [kg/m^3] \nonumber\\
  \Gamma &= \mbox{ domain boundary. }\nonumber
\end{align}

The second type or Neumann type boundary condition describes an impermeable 
boundary

\begin{align}
  \theta D_{ij}\frac{\partial C(x,y,z,t)}{\partial x_j}\hat{i} &= 0 \hspace{1mm}\mbox{ for } (x,y,z) \in \Gamma
  \intertext{where}
  \theta&= \mbox{ porosity } [-]\nonumber\\
  D_{ij} &= \mbox{ diffusion coefficient tensor component } [m^2/s]\nonumber.
\end{align}

The third, Cauchy, type describes a combination of the Dirichlet and Neumann 
type conditions, defining both a concentration at a boundary and a flux at that 
boundary, 

\begin{align}
  -\theta D_{ij}\frac{\partial C}{\partial x_j}\hat{i} + q_iC\hat{i} &= q_iC_0\hat{i}.
  \intertext{where}
  q_i\hat{i} &= \mbox{ outward fluid flux } [m/s]\nonumber\\
  \hat{i} &= \mbox{unit vector normal to the surface } [-]\nonumber\\
  C_0 &= \mbox{concentration of the fluid at the boundary } [kg/m^3].\nonumber
\end{align}

For a Cauchy boundary condition, each nuclide transport model must represent the 
solute concentration flux at the outer boundary in units of $[kg/m^{2}/s]$. In 
terms of the solute mass flux, the mass balance equation is 

\begin{align}
m_k^{n+1} &= m_k^0 + S_A\sum_{i=0}^n\left[ \sum_{j=0}^{children}\left( 
\vec{J}_{j,k}^i - \vec{J}_{k,l}^i\right)(t_n-t_{n-1})\right]
\intertext{where }
\vec{J}_{j,k} &= \mbox{solute flux between j and k at time }t_i [kg/m^{2}/s]
\label{cauchy_balance}
\end{align}


Each component must also provide a pure source term, a mass transfer per unit 
time,

\begin{align}
\dot{m}(x,y,z,t) &= \dot{m}_0(x,y,z,t)
\intertext{where}
\dot{m} &= \mbox{ mass transfer across the boundary }[kg/s].\nonumber
\label{source}
\end{align}

For the central source cell, which has no children

\begin{align}
m_0^{n+1} = m_0^0 - \sum_{i=0}^n m_{0,1}^n.
\end{align}

For a situation as in \Cyclus, with discrete timesteps, $t_n$, this becomes
the available source term at the boundary 
between cell $i$ and cell $j$ at time $t_n$ is $m_{i,j}^n$.

%In cylindrical coordinates, these boundary conditions are applied to the 
%advection dispersion equation in cylindrical coordinates 
%\cite{leij_analytical_1982}: 
%
%\begin{align}
%R \frac{\partial C}{\partial t} &= D_i \frac{\partial^2 C}{\partial x_i^2} - v 
%\frac{\partial C }{\partial x_i} \frac{D_r }{r} \frac{\partial }{\partial 
%r}\left[r\frac{\partial C}{\partial r}\right]
%\intertext{where}
%R &= \mbox{ retardation factor }[-]\nonumber\\
%C &= \mbox{ solute concentration } [kg/m^3]\nonumber\\
%t &= \mbox{ time } [s]\nonumber\\
%\hat{x}_i &= \mbox{ direction of flow } [-]\nonumber\\
%v &= \mbox{ pore water velocity } [m/s].\nonumber
%\label{ade_cyl}
%\end{align}
%
%Unfortunately, in the above, the direction of flow must be perpendicular to the 
%radial direction, which is not the situation we'd like to imagine.

%\frac{\partial }{\partial }
%<++> &= \mbox{ <++> } [<++>]\nonumber\\

