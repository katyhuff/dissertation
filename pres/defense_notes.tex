%        File: defense_notes.tex
%     Created: Tue Jul 30 03:00 PM 2013 C
% Last Change: Tue Jul 30 03:00 PM 2013 C
%
\documentclass[letterpaper]{article}
\usepackage[top=1.0in,bottom=1.0in,left=1.0in,right=1.0in]{geometry}
\usepackage{verbatim}
\usepackage{amssymb}
\usepackage{graphicx}
\usepackage{longtable}
\usepackage{amsfonts}
\usepackage{amsmath}
\usepackage[usenames]{color}
\usepackage[
naturalnames = true, 
colorlinks = true, 
linkcolor = Black,
anchorcolor = Black,
citecolor = Black,
menucolor = Black,
urlcolor = Blue
]{hyperref}
\def\thesection       {\arabic{section}}
\def\thesubsection     {\thesection.\alph{subsection}}

\author{K. Huff
\\ \href{mailto:khuff@cae.wisc.edu}{\texttt{khuff@cae.wisc.edu}}
}

\date{}
\title{Defense Notes}
\begin{document}
\maketitle

\textbf{Title}


\textbf{Outline}


\section{Motivation}

\textbf{Top Level Fuel Cycle Simulators}


\subsection{Future Fuel Cycle Options}
\textbf{Future Fuel Cycle Options}



\subsection{Geologic Disposal Options}
\textbf{Disposal Geology Options Considered}



\textbf{Repository Components}



\subsection{Fuel Cycle Simulator Capabilities}
\textbf{Need for an Integrated Repository Model}


\textbf{Outline}

\section{Modeling Paradigm}
\textbf{Cyder Paradigm: Modularity}
A modular repository framework facilitates
\begin{itemize}
\item interchangable subcomponents
\item and simulations with varying levels of detail.
\end{itemize}
Integration with a fuel cycle simulator facilitates
\begin{itemize}
\item analysis of feedback effects upon the fuel cycle
\item and fuel cycle impacts on disposal system performance.
\end{itemize}

\section{Demonstration}
\section{Conclusion}
\textbf{<++>}
\textbf{<++>}
\textbf{<++>}

An integrated repository performance model enables the analysis of global fuel
cycle metrics concerned with material routing, intermediate storage time, and
repository performance. Post-processed analysis of these metrics can neglect
feedback effects of geologic disposal capacity constraints and repository
performance within the fuel cycle system.


\pagebreak
\bibliographystyle{ieeetr}
\bibliography{paper}
\end{document}


