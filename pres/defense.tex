%        File: defense.tex
%     Created: Sun May 5 10:00 PM 2013 C
%


%\documentclass[11pt,handout]{beamer}
\documentclass[9pt]{beamer}
\usetheme[white]{Wisconsin}
%\title[short title]{long title}
\title[ Thesis Defense ]{ An Integrated Used Fuel Disposition and Generic Repository Model for Fuel Cycle Analysis }
%\subtitle[short subtitle]{long subtitle}
\subtitle[PhD Defense]{PhD Defense}
%\author[short name]{long name}
\author[Kathryn D. Huff]{Kathryn D. Huff}
%\date[short date]{long date}
\date[5.17.2013]{May 17, 2013}
%\institution[short name]{long name}
\institute[UW-Madison]{University of Wisconsin-Madison}

%\usepackage{bbding}
\usepackage{amsfonts}
\usepackage{amsmath}
\usepackage{graphicx}
\usepackage{subfigure}
\usepackage{booktabs} % nice rules for tables
\usepackage{microtype} % if using PDF
\usepackage{bigints}
\newcommand{\units}[1] {\:\text{#1}}%
\newcommand{\SN}{S$_N$}%{S$_\text{N}$}%{$S_N$}%
\DeclareMathOperator{\erf}{erf}

%page numbers
\setbeamertemplate{footline}[page number]
%Those icons in the references are terrible looking
\setbeamertemplate{bibliography item}[text]
\begin{document}
%%%%%%%%%%%%%%%%%%%%%%%%%%%%%%%%%%%%%%%%%%%%%%%%%%%%%%%%%%%%%
%% From uw-beamer Here's a handy bit of code to place at 
%% the beginning of your presentation (after \begin{document}):
\newcommand*{\alphabet}{ABCDEFGHIJKLMNOPQRSTUVWXYZabcdefghijklmnopqrstuvwxyz}
\newlength{\highlightheight}
\newlength{\highlightdepth}
\newlength{\highlightmargin}
\setlength{\highlightmargin}{2pt}
\settoheight{\highlightheight}{\alphabet}
\settodepth{\highlightdepth}{\alphabet}
\addtolength{\highlightheight}{\highlightmargin}
\addtolength{\highlightdepth}{\highlightmargin}
\addtolength{\highlightheight}{\highlightdepth}
\newcommand*{\Highlight}{\rlap{\textcolor{HighlightBackground}{\rule[-\highlightdepth]{\linewidth}{\highlightheight}}}}
%%%%%%%%%%%%%%%%%%%%%%%%%%%%%%%%%%%%%%%%%%%%%%%%%%%%%%%%%%%%%
%%--------------------------------%%
\frame{
  \titlepage
}

\section{Motivation}
%%--------------------------------%%
\begin{frame}
  \frametitle{Outline}
  \tableofcontents[currentsection]
\end{frame}

\subsection{Future Fuel Cycle Options}

% DOE is interested in many fuel cycle options. 
% DOE is considering 3 main options, each of which pose different waste 
% management challenges . . .

As the United States and other nations seek to develop technologies and strategies to support a 
sustainable future for nuclear energy, various fuel cycle strategies and 
corresponding disposal system options are being considered.  Specifically, the 
domestic fuel cycle option space under current consideration is described in 
terms of three distinct fuel cycle categories with the monikers Once Through, 
Full Recycle, and Modified Open. Each category presents unique disposal system 
design challenges. Systems analyses for evaluating these options must 
be undertaken in order to inform a national decision to deploy a comprehensive 
fuel cycle system by 2050 \cite{doe_nuclear_2010}. 

% Once through presents capacity issues . . . 

The Once-Through Cycle category includes fuel cycles similar to the fuel cycle 
currently deployed in the United States, utilizing light water reactors and 
direct disposal of spent nuclear fuel in a geologic repository.
Such fuel cycles neglect reprocessing and present challenges associated with 
high volumes of minimally treated spent fuel streams.  In in a geologic 
repository a business as usual 
scenario, conventional power reactors comprise the majority of nuclear energy 
production. 
Calculations from the Electric Power Research Institute corroborated by 
the \gls{US} \gls{DOE} in 2008 indicate that without an increase in the statutory 
capacity limit of the \gls{YMR}, continuation of the current Once Through fuel 
cycle will generate a volume of spent fuel that will necessitate
the siting of an additional federal geologic repository to accommodate spent 
fuel \cite{kessler_room_2006, doe_report_2008}. 

% Full Recycle presents the issue of variable waste streams. . .

A Full Recycle option, on the other hand, requires the research, development, 
and deployment of partitioning, transmutation, and advanced reactor technology 
for the reprocessing of used nuclear fuel.  In this scheme, conventional 
once-through reactors will be phased out in favor of advanced transmutation 
technologies. All fuel in the Full Recycle strategy will be 
reprocessed using an accelerator driven system or by 
cycling through an advanced fast reactor. Such fuel may undergo partitioning, 
the losses from which will require waste treatment and ultimate disposal in a 
repository. Thus, a repository under the Full Recycle scenario must support
a waste stream composition that is highly variable during transition periods as 
well as myriad waste forms and packaging associated with isolation of differing 
waste streams.

% Modified Open presents both problems. . . 

Finally, the Modified Open Cycle category of options includes a variety of fuel 
cycle options that fall between once through and fully closed. Advanced fuel 
cycles such as deep burn and small modular reactors will be considered as will 
partial recycle options.  Partitioning and reprocessing strategies, however, 
will be limited to simplified chemical separations and volatilization. This 
scheme presents a dual challenge in which spent fuel volumes 
and composition will both vary dramatically among various possibilities within 
this scheme \cite{doe_nuclear_2010} .

% Various waste streams require various WFs and WPs 

Clearly, the waste streams resulting from potential fuel cycles present an 
array of corresponding waste disposition, packaging, and engineered barrier 
system options. Differing spent fuel composition, partitioning, transmutation, 
and chemical processing decisions upstream in the fuel cycle may demand differing 
natural and engineered barrier requirements during disposal. The 
capability to model thermal and radionuclide transport phenomena of 
arbitrary isotopic compositions is therefore required. This work has produced a 
disposal system simulator that meets this need. 


\subsection{Geologic Disposal Concept Options}


%%----------------------------------------%%
\begin{frame}[ctb!]
  \frametitle{Repository Components}
\footnotesize{
  \input{skb_fig}
}
\end{frame}

\begin{frame}
  \frametitle{Repository Layouts}

  \begin{minipage}{0.49\textwidth}
    \begin{figure}[h!]
      \includegraphics[width=0.75\textwidth]{./images/boreholes.eps}
    \end{figure}
    \begin{figure}[h!]
      \includegraphics[width=0.75\textwidth]{./images/vertical.eps}
    \end{figure}
  \end{minipage}
  \hspace{0.01cm}
  \begin{minipage}{0.49\textwidth}
    \begin{figure}[h!]
      \includegraphics[width=0.8\textwidth]{./images/horizontal.eps}
    \end{figure}
    \begin{figure}[h!]
      \includegraphics[width=0.8\textwidth]{./images/alcoves.eps}
    \end{figure}
  \end{minipage}

\end{frame}



\begin{frame}[ctb!]
  \frametitle{Disposal Geology Options Considered}
   \begin{minipage}{0.44\textwidth}
     \begin{figure}[h!]
         \includegraphics[width=0.7\textwidth]{./images/saltNewScientist.eps}
         \caption{U.S. Salt Deposits, ref. \cite{newscientist_where_2011}.}
     \end{figure}
     \begin{figure}[h!]
         \includegraphics[width=0.7\textwidth]{./images/clayGonzales.eps}
         \caption{U.S. Clay Deposits, ref. \cite{gonzales_shales_1985}.}
     \end{figure}
   \end{minipage}
   \begin{minipage}{0.44\textwidth}
     \begin{figure}[h!]
         \includegraphics[width=0.7\textwidth]{./images/boreholeNewScientist.eps}
         \caption{U.S. Crystalline Basement, ref.  \cite{newscientist_where_2011}.}
     \end{figure}
     \begin{figure}[h!]
         \includegraphics[width=0.7\textwidth]{./images/graniteBush.eps}
         \caption{U.S. Granite Beds, ref. \cite{bush_economic_1976}.}
     \end{figure}
   \end{minipage}
\end{frame}

\subsection{Fuel Cycle Simulator Capabilities}

\section{Geologic Repository Performance}
\subsection{Metrics}
\include{performance_metrics}
\subsection{Thermal Loading}
\include{performance_thermal}
\subsection{Radionuclide Transport and Release}
\include{performance_release}

\section{Modeling Paradigm}
\subsection{Cyclus Modeling Paradigm}
\subsection{Cyder Modeling Paradigm}


\section{Abstracted Models}
\subsection{Thermal Transport in Cyder}
\chapter{Thermal Transport Computational Models}\label{ch:thermal_models}
The mass transfer interfaces between the mass balance models are essential to 
the understanding of the \Cyder paradigm.  

In groundwater transport, contaminants are transported by dispersion and 
advection. It is customary to define the combination of molecular diffusion and 
mechanical mixing as the dispersion tensor, $D$, such that, for a conservative 
solute (infinitely soluble and non-sorbing), the mass conservation equation 
becomes \cite{schwartz_fundamentals_2004, wang_introduction_1982, 
van_genuchten_analytical_1982}:

     \begin{align}
      J &= J_{dis} + J_{adv}\nonumber
      \intertext{where}
      J_{dis} &= \mbox{ Total Dispersive Mass Flux }[kg/m^2/s]\nonumber\\
      &= -\theta(D_{mdis} + \tau D_m)\nabla C \nonumber\\ 
      &= -\theta D\nabla C \nonumber\\
      J_{adv} &= \mbox{ Advective Mass Flux }[kg/m^2/s]\nonumber\\
      &= \theta vC\nonumber\\
      \tau &= \mbox{ Tortuosity }[-] \nonumber\\
      \theta &= \mbox{ Porosity }[-] \nonumber\\
      D_m &= \mbox{ Molecular diffusion coefficient }[m^2/s]\nonumber\\
      D_{mdis} &= \mbox{ Coefficient of mechanical dispersivity}[m^2/s]\nonumber\\
      D &= \mbox{ Effective Dispersion Coefficient }[m^2/s]\nonumber\\
      C &= \mbox{ Concentration }[kg/m^3]\nonumber\\
      v &= \mbox{ Fluid Velocity in the medium }[m/s].\nonumber
    \end{align}

For uniform flow in $\hat{k}$, 
    \begin{align}
      J &=\left(-\theta D_{xx} \frac{\partial C}{\partial x}
             \right)\hat{\imath}
             + \left( -\theta D_{yy} \frac{\partial C}{\partial y}
            \right)\hat{\jmath}
            + \left( -\theta D_{zz} \frac{\partial C}{\partial z}
             + \theta v_zC 
            \right)\hat{k}.
      \label{unidirflow}
    \end{align}

Solutions to this equation can be categorized by their boundary conditions.  
Those boundary conditions serve as the interfaces between components in the 
\Cyder library of nuclide transport models by way of advective, dispersive, 
coupled, and fixed fluxes.  This is supported by implementation in which 
vertical advective velocity is uniform throughout the system and in which 
parameters such as the dispersion coefficient are known for each component. 


<pphw>

  introduce 2 different mass transfer modes

  explicit: a model is chosen to represent the mass transfer and the sink 
  inventory is updated based on this transfer rate (I think these are only used 
  between a pair of 0-D models [KDH note: all components allow this on their 
  external boundary. Only the 0D models also allow this on their inner 
  boundary.]) introduce overall concept of combined advective and diffusive mass transfer

  implicit: a sink inventory is updated based on its distribution model, and 
  mass is transferred to accomplish the change in inventory


\subsection{Thermal Capacity Approximation Methodology}\label{sec:capacity}

\subsection{Lumped Parameter Thermal Transport Computational 
Model}\label{sec:lumped_thermal}


\subsection{Radionuclide Transport in Cyder}
\section{Radionuclide Transport In \textsc{Cyder}}\label{sec:nuclide_models}

The mass transfer interfaces between the mass balance models are essential to 
the understanding of the \Cyder paradigm.  

In groundwater transport, contaminants are transported by dispersion and 
advection. It is customary to define the combination of molecular diffusion and 
mechanical mixing as the dispersion tensor, $D$, such that, for a conservative 
solute (infinitely soluble and non-sorbing), the mass conservation equation 
becomes \cite{schwartz_fundamentals_2004, wang_introduction_1982, 
van_genuchten_analytical_1982}:

     \begin{align}
      J &= J_{dis} + J_{adv}\nonumber
      \intertext{where}
      J_{dis} &= \mbox{ Total Dispersive Mass Flux }[kg/m^2/s]\nonumber\\
      &= -\theta(D_{mdis} + \tau D_m)\nabla C \nonumber\\ 
      &= -\theta D\nabla C \nonumber\\
      J_{adv} &= \mbox{ Advective Mass Flux }[kg/m^2/s]\nonumber\\
      &= \theta vC\nonumber\\
      \tau &= \mbox{ Tortuosity }[-] \nonumber\\
      \theta &= \mbox{ Porosity }[-] \nonumber\\
      D_m &= \mbox{ Molecular diffusion coefficient }[m^2/s]\nonumber\\
      D_{mdis} &= \mbox{ Coefficient of mechanical dispersivity}[m^2/s]\nonumber\\
      D &= \mbox{ Effective Dispersion Coefficient }[m^2/s]\nonumber\\
      C &= \mbox{ Concentration }[kg/m^3]\nonumber\\
      v &= \mbox{ Fluid Velocity in the medium }[m/s].\nonumber
    \end{align}

For uniform flow in $\hat{k}$, 
    \begin{align}
      J &=\left(-\theta D_{xx} \frac{\partial C}{\partial x}
             \right)\hat{\imath}
             + \left( -\theta D_{yy} \frac{\partial C}{\partial y}
            \right)\hat{\jmath}
            + \left( -\theta D_{zz} \frac{\partial C}{\partial z}
             + \theta v_zC 
            \right)\hat{k}.
      \label{unidirflow}
    \end{align}

Solutions to this equation can be categorized by their boundary conditions.  
Those boundary conditions serve as the interfaces between components in the 
\Cyder library of nuclide transport models by way of advective, dispersive, 
coupled, and fixed fluxes.  This is supported by implementation in which 
vertical advective velocity is uniform throughout the system and in which 
parameters such as the dispersion coefficient are known for each component. 


<pphw>

  introduce 2 different mass transfer modes

  explicit: a model is chosen to represent the mass transfer and the sink 
  inventory is updated based on this transfer rate (I think these are only used 
  between a pair of 0-D models [KDH note: all components allow this on their 
  external boundary. Only the 0D models also allow this on their inner 
  boundary.]) introduce overall concept of combined advective and diffusive mass transfer

  implicit: a sink inventory is updated based on its distribution model, and 
  mass is transferred to accomplish the change in inventory


\subsection{Degradation Rate Radionuclide Transport Model}\label{sec:deg_rate}
The degradation rate model, simulating the fractional degradation of the 
material containment properties, is the simplest of implemented models and is most 
appropriate for simplistic modeling of a degrading barrier volume. The fundamental 
concept is depicted in Figure \ref{fig:deg_volumes}.

\begin{figure}[h!]
  \begin{center}
    \def\svgwidth{.7\textwidth}
    \input{./images/deg_volumes.eps_tex}
  \end{center}
  \caption[Constituents of a Degradation Rate Control Volume]{The control volume contains an 
  intact volume $V_i$ and a degraded volume, $V_d$. Contaminants in $V_d$ are 
  available for transport, while contaminants in $V_i$ are contained.}
  \label{fig:deg_volumes}
\end{figure}



The materials that constitute the engineered barriers in a 
repository environment degrade over time. The implemented model of this nuclide 
release behavior is based solely on a fractional degradation rate. 
The degraded volume is a simple fraction, $d$, of the total volume, $V_T$, such 
that 
\begin{align}
V_T &= V_i + V_d
\label{deg_volumes}
\intertext{where}
V_d(t) &= d(t)V_T\nonumber\\
V_i(t) &= (1-d(t))V_T\nonumber\\
V_T &= \mbox{ total volume }[m^3]\nonumber\\
V_i(t) &= \mbox{ intact volume at time t }[m^3]\nonumber\\
V_d(t) &= \mbox{ degraded volume at time t }[m^3]\nonumber
\intertext{and}
d(t) &= \mbox{ the fraction that has been degraded by time t }[-].\nonumber
\end{align}


\subsubsection{Calculation of Mass Transfer}
In this model, the contaminants in the degraded fraction of the control volume 
are available to adjacent components. The available contaminants
$m_{ij}(t)$, at the boundary between cell $i$ to cell $j$ at time $t$ are thus

\begin{align}
\dot{m}_{ij}(t) &= f_im_i(t)
\label{deg_rate_source_cont}
\intertext{where}
\dot{m}_{ij} &= \mbox{ the rate of mass transfer from i to j }[kg/s]\nonumber\\
f_i &= \mbox{ degradation rate function in cell i }[1/s] \nonumber\\
m_i &= \mbox{ mass in cell i }[kg] \nonumber\\
t &= \mbox{ time  }[s].\nonumber
\end{align}

For a situation as in \Cyder and \Cyclus, with discrete time steps, the time steps are 
assumed to be small enough to assume a constant rate $m_{ij}$ over the course 
of the time step. This model incorporates the source term made available on the 
inner boundary into its available mass and defines the resulting boundary 
conditions at the outer boundary as solely a function of the degradation rate 
of that component.  The mass transferred between discrete times $t_{n-1}$ and 
$t_n$ is thus a simple linear function of the transfer rate in 
\eqref{deg_rate_source_cont}, 

\begin{align}
m_{ij}(t_n) &= \int_{t_{n-1}}^{t_n}\dot{m}_{ij}(t')dt' \nonumber\\
         &= f_im_i(t_{n-1})\left[t_n - t_{n-1}\right].
\label{deg_rate_source_discrete}
\end{align}

\subsubsection{Boundary Interfaces}
\label{sec:dr_bc}
The mass $m_{f}(t_n)$ is the source term at the outer boundary, 
\begin{align}
  \mathcal{S}_j(t_n) &= \mbox{ fixed source term at }r_j [kg]\nonumber\\ 
                     &= m_{f}(t_n).
\end{align}
The concentration boundary condition must also be defined at the outer boundary 
to support parent components that utilize the Dirichlet boundary condition. For 
the degradation rate model, which incorporates no diffusion or advection, the 
concentration, $C_j$ at $r_j$, the boundary between cells $j$ and $k$, is the average 
concentration in the saturated pore volume,

\begin{align}
\mathcal{D}_j(t_n) &= \mbox{ fixed concentration from component j at }t_n [kg/m^3].\nonumber\\ 
                   &= C_{j}(t_n)\nonumber\\ 
                   &= C_{df}\nonumber\\
                   &= \frac{m_{d}(t_n)}{V_{d}(t_n)}\\
\label{deg_rate_dirichlet}\\
&= \frac{\mbox{ solute mass in degraded fluid in cell j }}{\mbox{ degraded fluid volume in cell j}}.\nonumber 
\end{align}

To support parent components that utilize the Neumann boundary condition, the 
concentration gradient can be found if the concentration and the 
radial midpoint of the external component, $k$, are specified.

\begin{align}
\mathcal{N}_j(t_n) &= \mbox{ fixed concentration gradient from component j at }t_n [kg/m^3/s]\nonumber\\
                   &= \frac{dC(t_n)}{dr}\Bigg|_{r=r_j}\nonumber\\ 
                   &= \frac{C_k(r_{k-1/2},t_{n-1}) - C_j(r_{j-1/2}, t_n)}{r_{k-1/2} - r_{j-1/2}}
\label{neumann_dr}
\intertext{where}
r_{j-1/2} &= r_{j} - \frac{r_{j} - r_i}{2}\nonumber\\
r_{k-1/2} &= r_{k} - \frac{r_{k} - r_j}{2}.\nonumber
\end{align}


To support parent components that utilize the Cauchy boundary condition, the 
degradation rate model assumes that the fluid velocity is constant across the cell 
as is the concentration. Thus, 

\begin{align}
\mathcal{C}_j(t_n) &= \mbox{ fixed concentration flux from component j at }t_n [kg/m^2/s]\nonumber\\
                   &= -D\frac{\partial C(t_n)}{\partial r}\Bigg|_{r=r_j} + v_zC(t_n)\Bigg|_{r=r_j} \\
\label{deg_rate_cauchy}
\end{align}

%<++> &= \mbox{ <++> }[<++>] \nonumber\\


%\section{Mixed Cell Volume Radionuclide Model}\label{sec:mixed_cell}

%\subsection{Lumped Parameter Thermal Transport Computational 
Model}\label{sec:lumped_thermal}

%\subsection{One Dimensional Permeable Porous Medium Radionuclide Transport 
Model}\label{sec:one_dim_ppm}
Finally, abstraction results informed modifications to the implementation of an 
analytic solution to the one dimensional advection-dispersion equation with 
finite domain and Cauchy and Neumann boundary conditions at the inner and outer 
boundaries, respectively. 

Various solutions to the advection dispersion equation  
\eqref{unidirflow} have been published for both the first and third types of 
boundary conditions. The third, Cauchy type, is mass conservative, and will be 
the primary kind of boundary condition used at the source for this model.

The conceptual model in Figure \ref{fig:1dinf} represents solute transport in 
one dimension with unidirectional flow upward (a conservative assumption) and a 
semi-infinite boundary condition in the positive flow direction. The solution is 
given (Leij et. al, \cite{leij_analytical_1991}) and described below.  

\begin{figure}[h!]
  \begin{center}
    \def\svgwidth{.5\textwidth}
    \input{./chapters/methodology/nuclide_models/one_dim_ppm/1dinf.eps_tex}
  \end{center}
  \caption[1D semi-infinite advection dispersion solution.]{A one dimensional, 
  semi-infinite, unidirectional flow solution with Cauchy and Neumann boundary 
conditions}
  \label{fig:1dinf}
\end{figure}

For the boundary conditions, 
\begin{align}
  -D \frac{\partial C}{\partial z}\big|_{z=0} + v_zc &= \begin{cases}
    v_zC_0  &  \left( 0<t<t_0 \right)\\
    0  &  \left( t>t_0 \right)\\
  \end{cases},\\
  \frac{\partial C}{\partial z}\big|_{z=\infty} &= 0
  \intertext{and the initial condition,}
  C(z,0) &= C_i,
  \label{1dinfBC}
  \intertext{the solution is given as }
  C(z,t) &= \begin{cases} 
  \left(C_0 - \frac{\gamma}{\mu}\right)A(z,t) + B(z,t) & 0<t\le t_0\\
  \left(C_0 - \frac{\gamma}{\mu}\right)A(z,t) + B(z,t) - C_0A(z,t-t_0)& t\ge t_0
  \end{cases}
\intertext{where}
  \gamma &= \mbox{ zero order source term in the fluid }[kg/m^3]\nonumber\\
  \mu &= \mbox{first order general decay constant }[s^{-1}].\nonumber
\end{align}


For the vertical flow coordinate system, $A$ and $B$ are defined as
\begin{align}
A(z,t) &= 
frac{v}{v+u}e^{\frac{(v-u)z}{2D}}\erfc{\left[\frac{Rz-ut}{2\sqrt{DRt}}\right]} \nonumber\\
&+ \frac{v}{v-u}e^{\frac{(v+u)z}{2D}}\erfc{\left[\frac{Rz+ut}{2\sqrt{DRt}}\right]} \nonumber\\ 
&+ \frac{v^2}{2\mu D} e^{\left(\frac{vz}{D} - \frac{\mu t}{R}\right)}\erfc{\left[\frac{Rz+vt}{2\sqrt{DRt}}\right]}\\
B(z,t) &= 
\left(\frac{\gamma}{\mu} - C_i\right)\Bigg\{ \frac{1}{2}\erfc{\left[\frac{Rz - vt}{2\sqrt{DRt}}\right]}\nonumber\\
&+ \left(\frac{v^2t}{\pi RD}\right)^{1/2}e^{-\frac{(Rz-vt)^2}{4DRt}}\nonumber\\ 
&- \frac{1}{2} \left(1+\frac{vz}{D} + \frac{v^2t}{DR}\right) e^{\frac{vz}{D}}\erfc{\left[\frac{Rz+vt}{2\sqrt{DRt}}\right]}\Bigg\}\nonumber\\
\intertext{where}
R &= \mbox{Retardation factor }[-].
\end{align}

We make the simplification that, in Cyclus, radioactive decay is handled external 
to the components, so $\mu$, the constant of decay goes to zero. Similarly, we 
take the zero order source term to be zero, as no source term should emerge 
without decay or external injection. 

\begin{align}
  \intertext{which, for no $\lambda$ first order production becomes}
  &= \frac{C_0}{4}\int_0^t\frac{v}{R}\Lambda_3(\tau)\Gamma_2(\tau)d\tau
\end{align}

\begin{align}
  \Lambda_3(\tau) &= e^{-\frac{\mu\tau}{R}}\Bigg[\sqrt{\frac{R}{\pi D_z\tau}}e^{-\frac{(Rz-v\tau)^2}{4RD_z\tau}} - 
    \frac{v}{2D_z}e^\frac{vz}{D_z}\erfc{\frac{Rz+v\tau}{\sqrt{4RD_z\tau}}}\Bigg]
    \label{Lambda_3}
  \intertext{and}
  \Gamma_2(\tau) &= 
      \Bigg[ \erfc{\frac{x-a}{\sqrt{\frac{4D_x\tau}{R}}} } - 
             \erfc{\frac{x+a}{\sqrt{\frac{4D_x\tau}{R}}} } \Bigg]
      \Bigg[ \erfc{\frac{y-b}{\sqrt{\frac{4D_y\tau}{R}}} } -
             \erfc{\frac{y+b}{\sqrt{\frac{4D_y\tau}{R}}} } \Bigg]. 
      \label{Gamma_2}
\end{align}

\begin{align}
  C(z,t)&= \frac{C_0}{2}\int_0^t\frac{v}{R}
  \Bigg[\sqrt{\frac{R}{\pi D_z\tau}}e^{-\frac{(Rz-v\tau)^2}{4RD_z\tau}} -
    \frac{v}{2D_z}e^\frac{vz}{D_z}\erfc{\frac{Rz+v\tau}{\sqrt{4RD_z\tau}}}\Bigg]
    \erfc{\frac{1}{\sqrt{\frac{4D_{xy}\tau}{R}}} }^2
  d\tau .
  \label{simple_leij}
\end{align}
\subsubsection{Calculating Mass Transfer}
The mass transfer rate relies on an algorithm similar to that of the 
LumpedNuclide model. However, the integral, by virtue of not having a closed 
form solution, must be evaluated numerically. 

\subsubsection{Boundary Conditions}
Again, the boundary conditions for the external boundary are found based on the 
mass contained in the cell by te same process as in the Degradation Rate model.
See equations \ldots 


%\subsection{Interfaces}
The interfaces between the models are essential to the understanding of the 
models themselves. The interfaces define boundary conditions in a number of 
forms based on information available internally to the component implementation. 

In a saturated, reducing environment, contaminants are transported by 
dispersion and advection. It is customary to define the combination 
of molecular diffusion and mechanical
mixing as the dispersion tensor, $D$, such that the mass conservation equation 
becomes \cite{schwartz_fundamentals_2004, wang_introduction_1982, van_genuchten_analytical_1982}:

    \begin{align}
      J &= J_{dis} + J_{adv}\nonumber\\
      &= -\theta(D_{mdis} + \tau D_m)\nabla C + \theta vC\nonumber\\ 
      &= -\theta D\nabla C + \theta vC \nonumber\\ 
      \intertext{which, for uniform flow in $\hat{k}$, is}
      &=\left(-\theta D_{xx} \frac{\partial C}{\partial x}
             \right)\hat{\imath}
             + \left( -\theta D_{yy} \frac{\partial C}{\partial y}
            \right)\hat{\jmath}
            + \left( -\theta D_{zz} \frac{\partial C}{\partial z}
             + \theta v_zC 
            \right)\hat{k},
      \label{unidirflow}
      \intertext{where}
      J_{dis} &= \mbox{ Total Dispersive Mass Flux }[kg/m^2/s]\nonumber\\
      J_{adv} &= \mbox{ Advective Mass Flux }[kg/m^2/s]\nonumber\\
      \tau &= \mbox{ Tortuosity }[-] \nonumber\\
      \theta &= \mbox{ Porosity }[\%] \nonumber\\
      D_m &= \mbox{ Molecular diffusion coefficient }[m^2/s]\nonumber\\
      D_{mdis} &= \mbox{ Coefficient of mechanical dispersivity}[m^2/s]\nonumber\\
      D &= \mbox{ Effective Dispersion Coefficient }[m^2/s]\nonumber\\
      C &= \mbox{ Concentration }[kg/m^3]\nonumber\\
      v &= \mbox{ Fluid Velocity in the medium }[m/s].\nonumber
    \end{align}

Solutions to this equation can be categorized by their boundary conditions and 
those boundary conditions serve as the interfaces between components in the 
\Cyder library of nuclide transport models.

  \begin{figure}[htp!]
    \begin{center}
      \def\svgwidth{\textwidth}
      \input{./chapters/methodology/nuclide_models/interfaces/flow.eps_tex}
    \end{center}
    \caption[\Cyder Component interfaces provide four Boundary Condition types.]{The boundaries between components (e.g., waste form and waste 
      package) are robust interfaces defined by boundary condition types.}
    \label{fig:flow}
  \end{figure}

In addition to a specified source term, the first, specified-head or Dirichlet type boundary conditions define a specified species 
concentration on some section of the boundary of the representative volume, 

    \begin{align}
      C(\vec{r},t) = C_0(\vec{r},t)\hspace{1mm}\mbox{ for } \vec{r} \in 
      \Gamma.
    \end{align}

The second type, specified-flow or Neumann type boundary conditions describe a full set of 
concentration gradients at the boundary of the domain,

    \begin{align}
      \frac{\partial C(\vec{r},t)}{\partial r} &= \theta D\vec{J}(t) \hspace{1mm}\mbox{ for } 
      \vec{r} \in \Gamma
      \intertext{where}
      \vec{r} &= \mbox{ position vector }\nonumber\\
      \Gamma &= \mbox{ domain boundary }\nonumber\\
      \vec{J}(t) &= \mbox{ solute mass flux } [kg/m^2\cdot s].\nonumber
    \end{align}
    

The third, head-dependent mixed boundary condition or Cauchy type, defines a solute 
flux along a boundary,

    \begin{align}
      -D\frac{\partial C}{\partial z} + v_zC &= v_zC(\vec{r},t) 
      \hspace{1mm}\mbox{ for }\vec{r} \in \Gamma.
    \end{align}  

The spatial concentration throughout the volume is sufficient to fully describe 
implementation of the following nuclide transport models within \Cyder. This is 
supported by the implementation in which vertical advective velocity is uniform 
throughout the system and in which parameters such as the dispersion coefficient 
are known for each component. Since this is the case in \Cyder, description of 
the Dirichlet condition is sufficient to fully define calculation of the Neumann 
and Cauchy type conditions.




\subsection{Implications For Waste Form Modeling}

Though the Waste Form Component can be modeled by any of the available 
NuclideModels, the Degradation Rate based or Mixed Cell radionuclide transport 
models are preferred for modeling of the Waste Form Component.  This is 
because, in repository performance assessments, waste form dissolution is 
typically modeled as instantaneous or rate based. Dissolution related release 
is historically modeled as congruent, solubility limited, or both, with some 
radionuclides becoming immediately accessible, and some tending to remain in 
the fuel matrix. 

\subsection{Implications For Waste Package Modeling}

Though the Waste Package Component can be modeled by any of the available 
NuclideModels, the simple Degradation Rate based model is strongly preferred.
Waste package time to failure is dependent on water contact and heat, 
but is historically modeled probabilistically, or at a constant rate.
Accordingly, waste package degradation in repository performance is either 
neglected entirely, instantaneous and complete (a delay before full release), 
or partial and constant (a constantly present hole in the package). 


\subsection{Implications For Buffer Modeling}

Diffusion is the primary mechanism for nuclide transport through the 
buffer Component of the repository system. While the buffer may degrade, the 
near field has historically been modeled in as much hydrologic detail as 
possible. For this reason, the Lumped Parameter or One Dimensional Permeable 
Porous Medium nuclide transport models are preferred.

\subsection{Implications for Geologic Environment Modeling}

In most of the saturated, low permeability environments being considered, 
diffusion is the primary mechanism for nuclide transport through the geologic 
medium Component of the repository system. While the near field may degrade, 
the far field should be modeled in detail if possible. For this reason, the 
One Dimensional Permeable Porous Medium nuclide transport 
model is preferred.


\section{Demonstration}
\subsection{Thermal Toy Cases}
\subsection{Radionuclide Toy Cases}
\subsection{Thermal Validation Cases}
\subsection{Radionuclide Validation Cases}

\chapter{Conclusions}\label{ch:conclusion}
\section{Contributions}

This work has provided a flexible code for rapid medium fidelity calculation of 
generic repository performance in the context of fuel cycle analysis.  Capable 
of thermal transport, hydrologic contaminant transport, and integration 
within a fuel cycle simulation code, \Cyder is the first of its kind.  

In addition to implementing fundamental modeling capabilities, \Cyder has been 
designed to accommodate the development of advanced capabilities in the future.

In this work, key conceptual components and modeling methods for geologic 
radioactive waste disposal were identified as part of a literature review, 
dominant physics of thermal and radionuclide transport were identified by 
conducting sensitivity analyses with detailed codes. Accordingly, a basic set 
of abstracted models were developed and implemented within the \Cyder code. 

A set of basic capabilities within the \Cyder library have been developed and 
validated and an assortment of advanced features, data, testing, and plotting 
capabilities are functional.  The \Cyder source code in which these models are 
implemented  is made freely available to interested researchers and potential 
model developers \cite{huff_cyder_2013}. In addition to the source code and 
supporting publications, the \Cyder code is well commented and produces 
clickable, browsable automated documentation with each build. That 
documentation is also available online.

The application programming interface to this software library is intentionally 
general, facilitating the incorporation of the models presented here within 
external software tools in need of a multicomponent disposal system simulator. 

Furthermore, this work contributes to an expanding ecosystem of computational 
models available for use with the \Cyclus fuel cycle simulator. This hydrologic 
nuclide transport library, by virtue of its capability to modularly integrate 
with the \Cyclus fuel cycle simulator has laid the foundation for integrated 
disposal option analysis in the context of fuel cycle options. 

\section{Suggested Future Work}
It is hoped that \Cyder will benefit from continued development and use. Future 
development efforts will likely be led by developer use cases, but are likely 
to include a number of advanced features that have the potential to extend the 
capabilities of this tool in significant ways. 

Initially, further validation of these models should include full benchmarks 
against the \gls{GDSM} results including biosphere conversion of the released 
source term.  Furthermore, thermal benchmarks against recent \gls{UFD} work for 
various design concepts would similarly improve the understanding of the range 
of validity for the thermal model. 

Thermal analysis in these results have been used to assess thermal performance 
of a repository after emplacement. However, dynamic, thermal capacity limited 
fuel cycle analyses concerning the variation of necessary cooling times among 
repository concepts and fuel cycles should be conducted using the capacity 
determination capability arrived at with this model.  

Additional advanced capabilities should include the incorporation of fracture 
enabled transport in a radionuclide transport model. This feature would improve 
analyses of geologic host media such as granite which exhibit significant 
cracking. Similarly, incorporation of a biosphere model in the far field would 
substantively benefit the calculation of fuel cycle metrics related to human and 
environmental effects and will support myriad expected use cases of the tool.

Additional radionuclide transport models, thermal transport models, and 
supporting data will enrich the capabilities of this code. 


%%--------------------------------%%
%%--------------------------------%%
\begin{frame}[allowframebreaks]
  \frametitle{References}
  \bibliographystyle{plain}
  {\footnotesize \bibliography{ne571} }

\end{frame}

%%--------------------------------%%




\end{document}



