%        File: defense_notes.tex
%     Created: Tue Jul 30 03:00 PM 2013 C
% Last Change: Tue Jul 30 03:00 PM 2013 C
%
\documentclass[letterpaper]{article}
\usepackage[top=1.0in,bottom=1.0in,left=1.0in,right=1.0in]{geometry}
\usepackage{verbatim}
\usepackage{amssymb}
\usepackage{graphicx}
\usepackage{longtable}
\usepackage{amsfonts}
\usepackage{amsmath}
\usepackage[usenames]{color}
\usepackage[
naturalnames = true, 
colorlinks = true, 
linkcolor = Black,
anchorcolor = Black,
citecolor = Black,
menucolor = Black,
urlcolor = Blue
]{hyperref}
\def\thesection       {\arabic{section}}
\def\thesubsection     {\thesection.\alph{subsection}}

\author{K. Huff
\\ \href{mailto:khuff@cae.wisc.edu}{\texttt{khuff@cae.wisc.edu}}
}

\date{}
\title{Defense Notes}
\begin{document}
\maketitle

\section*{Title}

Introduce yourself.

\section*{Outline}



\section*{Top Level Fuel Cycle Simulators}

Nuclear fuel cycle simulation seeks to provide a system level overview of the 
movement of materials and other quantities of interest among facilities as 
nuclear energy is produced. Using fuel cycle simulation tools, the potential 
impact of novel technologies and policies can be approximated. 

\Cyclus is one such simulation tool. The \Cyder tool, which is the primary 
subject of this work, represents the geologic repository, here.

\section*{Future Fuel Cycle Options}

A number of potential nuclear fuel cycles are being considered domestically. 

\section*{Disposal Geology Options Considered}

A number of potential nuclear waste repository geologies are being considered 
domestically. 

\section*{Repository Components}



\section*{Need for an Integrated Repository Model}

An integrated repository performance model enables the analysis of global fuel
cycle metrics concerned with material routing, intermediate storage time, and
repository performance. Post-processed analysis of these metrics can neglect
feedback effects of geologic disposal capacity constraints and repository
performance within the fuel cycle system.


\section*{Outline}


\section*{Waste Stream Acceptance}
The Cyder Facility dynamically accepts material from the coupled fuel
cycle simulation. The capacity decision is the interface at which feedbacks 
occur.

\section*{Waste Form Conditioning}

Waste conditioning is the process of packing a waste stream into an appropriate 
waste form.
In Cyder, discrete waste streams are conditioned into the appropriate
discrete waste form according to user-specified pairings.

\section*{Waste Packaging}
Waste packaging is the process of placing one or many waste forms into a 
containment package.
In Cyder, one or more waste forms are loaded into the appropriate waste
package according to user-specified pairings. 


\section*{Wate Package Emplacement}

Finally, the waste package is emplaced in a buffer component, which contains 
many other waste packages, spaced evenly in a grid. The grid is defined by the 
user input and depends on repository depth, z, waste package spacing, x, and 
tunnel spacing, y as in Figure 10.

\section*{Cyder Paradigm: Modularity}
A modular repository framework facilitates
\begin{itemize}
\item interchangable subcomponents
\item and simulations with varying levels of detail.
\end{itemize}
Integration with a fuel cycle simulator facilitates
\begin{itemize}
\item analysis of feedback effects upon the fuel cycle
\item and fuel cycle impacts on disposal system performance.
\end{itemize}

\section*{<++>}
\section*{<++>}
\section*{<++>}
\section*{<++>}
\section*{<++>}
\section*{<++>}


\pagebreak
\bibliographystyle{ieeetr}
\bibliography{paper}
\end{document}


