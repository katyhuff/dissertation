%        File: defense_notes.tex
%     Created: Tue Jul 30 03:00 PM 2013 C
% Last Change: Tue Jul 30 03:00 PM 2013 C
%
\documentclass[letterpaper]{article}
\usepackage[top=1.0in,bottom=1.0in,left=1.0in,right=1.0in]{geometry}
\usepackage{xspace}
\usepackage{verbatim}
\usepackage{amssymb}
\usepackage{graphicx}
\usepackage{longtable}
\usepackage{amsfonts}
\usepackage{amsmath}
\usepackage[usenames]{color}
\usepackage[
naturalnames = true, 
colorlinks = true, 
linkcolor = Black,
anchorcolor = Black,
citecolor = Black,
menucolor = Black,
urlcolor = Blue
]{hyperref}
\def\thesection       {\arabic{section}}
\def\thesubsection     {\thesection.\alph{subsection}}
\newcommand{\Cyder}{\textsc{Cyder}\xspace}
\newcommand{\Cyclus}{\textsc{Cyclus}\xspace}

\author{K. Huff
\\ \href{mailto:khuff@cae.wisc.edu}{\texttt{khuff@cae.wisc.edu}}
}

\date{}
\title{Defense Notes}
\begin{document}
\maketitle

\section*{Title}

Introduce yourself.

\section*{Outline}



\section*{Top Level Fuel Cycle Simulators}

Nuclear fuel cycle simulation seeks to provide a system level overview of the 
movement of materials and other quantities of interest among facilities as 
nuclear energy is produced. Using fuel cycle simulation tools, the potential 
impact of novel technologies and policies can be approximated. 

\Cyclus is one such simulation tool. The \Cyder tool, which is the primary 
subject of this work, represents the geologic repository, here.

\section*{Future Fuel Cycle Options}

A number of potential nuclear fuel cycles are being considered domestically. 

\section*{Disposal Geology Options Considered}

A number of potential nuclear waste repository geologies are being considered 
domestically. 

\section*{Repository Components}



\section*{Need for an Integrated Repository Model}

An integrated repository performance model enables the analysis of global fuel
cycle metrics concerned with material routing, intermediate storage time, and
repository performance. Post-processed analysis of these metrics can neglect
feedback effects of geologic disposal capacity constraints and repository
performance within the fuel cycle system.


\section*{Outline}


\section*{Waste Stream Acceptance}
The Cyder Facility dynamically accepts material from the coupled fuel
cycle simulation. The capacity decision is the interface at which feedbacks 
occur.

\section*{Waste Form Conditioning}

Waste conditioning is the process of packing a waste stream into an appropriate 
waste form.
In Cyder, discrete waste streams are conditioned into the appropriate
discrete waste form according to user-specified pairings.

\section*{Waste Packaging}
Waste packaging is the process of placing one or many waste forms into a 
containment package.
In Cyder, one or more waste forms are loaded into the appropriate waste
package according to user-specified pairings. 


\section*{Wate Package Emplacement}

Finally, the waste package is emplaced in a buffer component, which contains 
many other waste packages, spaced evenly in a grid. The grid is defined by the 
user input and depends on repository depth, z, waste package spacing, x, and 
tunnel spacing, y as in Figure 10.

\section*{Cyder Paradigm: Modularity}
A modular repository framework facilitates
\begin{itemize}
\item interchangable subcomponents
\item and simulations with varying levels of detail.
\end{itemize}
Integration with a fuel cycle simulator facilitates
\begin{itemize}
\item analysis of feedback effects upon the fuel cycle
\item and fuel cycle impacts on disposal system performance.
\end{itemize}

\section*{<++>}
\section*{<++>}
\section*{<++>}
\section*{<++>}
\section*{<++>}
\section*{<++>}


\pagebreak
\bibliographystyle{ieeetr}
\bibliography{paper}






\section*{Cyder Advective Diffusive Sensitivity}
Increased advection and increased diffusion lead to greater release. Also, when 
both are varied, a boundary between diffusive and advective
regimes can be seen. An example of these results are shown here.
Advection vs. Diffusion Sensitivity in Cyder. Dual advective velocity 
and reference diffusivity sensitivity for a non-sorbing, infinitely soluble 
nuclide.

  \section*{Thermal Base Case Demonstration}
A validation exercise comparing the combined scaling and  
superposition calculations demonstrates an average error of 1.1\% and a 
maximum error of 4.4\%.

This comparison of STC calculated thermal response from $Cm$ 
inventory per MTHM in 51GWd burnup UOX PWR fuel compares favorably with results 
from the semi-analytic model from LLNL.


  \section*{Thermal Base Case Demonstration}
Here, percent error is 
\begin{align}
\mbox{ percent error } &= 100\times\frac{\left|\Delta T_{LLNL} - \Delta 
T_{STC}\right|}{ \Delta T_{LLNL}}.
\end{align}

\section*{LLNL Model Waste Package Spacing Sensitivity}
  Figure \ref{fig:Cm242spacing_sens} shows the trend in which increased waste 
  package spacing of a medium decreases areal thermal energy 
  deposition in the near field.
  $K_{th}$ Sensitivity to $s$.
  
  Increased waste package 
  spacing decreases areal thermal energy deposition 
  (here represented by STC) in the near field (here $r_{calc} = 0.5m$).


\section*{LLNL Model Limiting Radius Sensitivity}
  The location of the limiting radius has a strong effect on the 
  waste package loading limit, for a fixed limiting temperat
  
  $K_{th}$ Sensitivity to $r_{lim}$.
  
  Increased limiting radius decreases thermal energy deposition contributing to 
  the thermal limit (here represented by STC).

\section*{Cyder Spacing and Limiting Radius Sensitivity}
  The thermal diffusivity was compared both with the 
  spacing between waste packages and the limiting radius. 

  $alpha_{th}$ vs. $r_{lim}$ Sensitivity in Cyder.

  Cyder results agree with those of the LLNL model. The importance of the 
  limiting radius decreases with increased $K_{th}$. The above example thermal 
  profile results from 10kg of $^{242}Cm$.

\section*{LLNL Model Thermal Conductivity Sensitivity}
By varying the thermal conductivity of the repository model from 0.1 to 4.5 
$[W\cdot m^{-1} \cdot K^{-1}]$, this sensitivity analysis succeeds in capturing 
the domain of thermal conductivities witnessed in high thermal conductivity 
salt deposits as well as low thermal conductivity clays.
$K_{th}$ Sensitivity for Low $\alpha_{th}$ in LLNL Model.
Increased thermal conductivity decreases thermal energy deposition 
(here represented by STC) in the near field (here $r_{calc} = 0.5m$).

\section*{Cyder Thermal Conductivity and Limiting Radius Sensitivity}

These figures validate the trend noted above that 
increased thermal conductivity of a medium decreases thermal energy deposition 
in the near field. Additionally, analysis with the \Cyder STC database 
demonstrates the way in which the importance of spacing and the importance of 
the limiting radius decrease with increasing $K_{th}$.

$K_{th}$ vs. $r_{lim}$ Sensitivity in Cyder
Cyder results agree with 
those of the LLNL model. The importance of the limiting radius decreases with 
increased $K_{th}$. The above example thermal profile results from 10kg of 
$^{242}Cm$.

\section*{Cyder Thermal Conductivity and Limiting Radius Sensitivity}

$K_{th}$ vs. Waste Package Spacing Sensitivity in Cyder.
Cyder results 
agree with 
those of the LLNL model. The importance of the limiting radius decreases with 
increased $K_{th}$. The above example thermal profile results from 10kg of 
$^{242}Cm$




\section*{LLNL Model Thermal Diffusivity Sensitivity}
  By varying the thermal diffusivity of the disposal system from $0.1-3\times 
  10^{-6} [m^2\cdot s^{-1}]$, this sensitivity analysis succeeds in capturing 
  the domain of 
  thermal diffusivities witnessed in high thermal diffusivity salt deposits as 
  well as low thermal diffusivity clays.
    Increased thermal diffusivity decreases thermal energy deposition (here represented by STC) in the near field (here $r_{calc} = 0.5m$).


\section*{Cyder Thermal Diffusivity and Conductivity Sensitivity}
$\alpha_{th}$ vs. $K_{th}$ Sensitivity in Cyder. Cyder trends agree
  with those of the LLNL model, in which increased thermal diffusivity results 
  in 
  decreased thermal depsoition in the near field. The above example thermal 
  profile results from 10kg of $^{242}Cm$.

\section*{Cyder Thermal Diffusivity and Limiting Radius Sensitivity}
  Further \Cyder analysis shows the importance of $K_{th}$ remains constant, 
  but 
  the importance of the limiting radius decreases with increasing 
  $\alpha_{th}$.
  $alpha_{th}$ vs. $r_{lim}$ Sensitivity in Cyder.
  Cyder trends agree with 
  those of the LLNL model. The importance of the limiting radius decreases with 
  increased $K_{th}$. The above example thermal profile results from 10kg of 
  $^{242}Cm$
\end{document}


