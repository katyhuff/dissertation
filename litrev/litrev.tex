\chapter{Literature Review}\label{ch:litrev}

The following literature review addresses five areas of current research
integral to the work at hand. The contribution of computational nuclear fuel
cycle simulation tools to sensitivity analyses of repository performance
metrics is first summarized. A review of analytical models of nuclide transport
follows, after which follows a review of analytical models of heat transport.
An overview of current detailed computational models, available data and
algorithms characterizing nuclide transport follows, including both standalone
and those incorporated into nuclear fuel cycle simulation tools. Finally, a
review of current computational models of heat transport in the waste disposal
system context is given.  Special focus is paid to the availability of
supporting data and algorithms informing geochemical and hydrogeological
transport on long time scales and in various geologies. 

\section{Repository Capabilities within Systems Analysis Tools}
\label{sec:SA_repos}

%%%%%%%%%%%%%%%%%%%%%%%%
% Systems Analysis Repository Capabilities
%%%%%%%%%%%%%%%%%%%%%%%% 
% The total system performance assessment is one type.Repository modules 
% incorporated into VISION and whatnot are another type.  Things to ask about 
% each of them include: % Which geologies do they model?  How long do they take 
% to run?  Are they proprietary?  How well validated are they?  Do they include 
% any notion of repository capacity, dose, heat?  Are they capable of dealing 
% with waste of varying compositions?  Are therewasteform models?  Is there 
% nuclide transport, source term estimate?  Heattransport?

%% This section comes from 2011 CFP Narrative 3067, only lightly edited %%

Current top-level simulators largely disregard the waste disposal phase of fuel
cycle analysis. Choosing instead to report metrics such as mass or volumes of
accumulated spent nuclear fuel, these analyses fail to address the impact of
those waste streams on the performance of the geologic disposal system. 
\cite{wilson_comparing_2011}  To fully inform the decision making process, 
metrics that depend on the performance of the geologic disposal system will be
necessary. 

A model for repository capacity was developed for the \gls{VISION} fuel cycle
simulator \cite{yacout_visionverifiable_2006} \cite{radel_repository_2007} and
recent efforts on the NUWASTE simulator \cite{ abkowitz_nuclear_2010} have made
some progress in addressing this deficiency, but despite a proliferation of
sophisticated fuel cycle simulators, similar efforts are lacking in this
regard. 



\subsection{NUWASTE} 

\gls{NUWASTE} is a Nuclear Waste Technical Review Board code that determines
many metrics about the fuel cycle according to various parameters.
\cite{abkowitz_nuclear_2010} This code tracks 65 isotopes within material
objects, discretely models individual shipping casks and incorporates cooling
time in both dry and wet intermediate storage facilities.

Though it tracks the number of assemblies through a simulation and their 
isotopic composition, \gls{NUWASTE} lacks waste package, waste form, source 
term, and dose calculations.

\subsection{VISION} 

\gls{VISION}'s repository model conducts decay calculations
and tracks upwards of 80 isotopes of interest in the nuclear fuel cycle.
\cite{yacout_visionverifiable_2006} While it calculates some basic information 
about waste package heat production and heat based repository capacity, discrete  
waste packages are not modeled and emplaced spent fuel composition is assumed to  
be spatially homogeneous. \cite{radel_simulation_2005}
\cite{boucher_international_2010}

\subsection{DANESS} 

\gls{DANESS} is developed at Argonne National Laboratory and
discretely models reactors within regional reactor parks. Material movement
is based on a fuel ordering paradigm and is written in a combination of
FortranIV and C, but is limited to a 10 reactor simulation and requires 
15 minutes for a 100 year simulation on a modern PC.  Input and output are 
in Microsoft Excel format and \gls{DANESS} relies on the proprietary IThink 
simulation platform. The code therefore has a secure license sensitive to 
export control and requiring explicit developer approval. 
\cite{yacout_daness_2011,van_den_durpel_daness:_2006} 

\subsection{COSI}

\gls{COSI} is a code from the CEA of France implemented in the Java programming 
language. It is capable of performing a large range of full fuel cycle 
scenarios. COSI6 is capable of making assessments of repository capacity as 
well as radiotoxicity and decay heat calculations of waste packages. 
\cite{boucher_international_2010}. Nuclide transport through the repository 
post-emplacement is, however, not calculated and the distribution liscence is 
very restrictive.


\subsection{NFCSim}

\gls{NFCSim} was implemented in the Java programming language by Erich Schneider 
and Los Alamos National Lab. Primarily a mass tracking code, the repository 
model was limited to a heat analysis. Uniquely, in NFCSim the transportation of 
each waste package was modeled discretely. \cite{schneider_nfcsim_2004}

\subsection{CAFCA}

\gls{CAFCA} is a fuel cycle code from MIT. While \gls{CAFCA} is capable of 
tracking processed fuel assemblies and isotopics, it does not calculate 
capacity estimations or conduct detailed nuclide or heat transport.  

\subsection{ORION} 

This code is a proprietary code developed at the United
Kingdom's National Nuclear Laboratory. While there is no heat-limited capacity 
model or calculation of nuclide transport, the material destined for the 
repository can be described with a number of mass indexed metrics such as activity 
$[Bq]$, radiotoxicity $[Sv]$,  toxic potential $[m^3]$, spontaneous neutron emission 
$[s^{-1}]$, and heat production $[W]$.



\subsection{OCRWM Yucca Mountain Total System Model}

The \gls{TSM} code, 
developed at \gls{OCRWM} is a very detailed model of the Yucca Mountain disposal 
system. It includes transportation issues and detailed emplacement timing and 
strategy models, but considers only the fuel cycle associated with the current  
U.S.  reactor fleet. Casks are modeled discretely and nuclide and heat transport 
are modeled in great detail .  This level of detail 
results in a dramatically extended run time. The \gls{TSM} model can only be 
run by its development team and runs a typical simulation, processing 70,000 
MTHM in 12-15 hours.  \cite{turner_discrete_2010} 

The TSM framework is based on the proprietary SimCAD platform. The simulation 
steps through times in 8 hour time steps during the waste cask transportation, 
processing and emplacement. This event based simulator is primarily focused on 
the operation stage of the Yucca Mountain repository, but is equipped with a 
thermal managment model which informs waste package emplacment.


\subsection{Repository Focused Fuel Cycle Analyses}

Focused fuel cycle sensitivity analyses emphasizing used fuel disposition and
waste management in the Yucca Mountain Repository (YMR) have been conducted by
Li, Piet, Wilson, and Ahn. With a focus on YMR capacity benefit, repository
performance metrics of interest for these analyses were heat, source term, and
more global envrionmental impact metrics.  Sensitivity analyses for other
geologies were conducted concerning repository concepts relevant to other
nations as well. See Table \ref{tab:red}.

\subsection{European RED-IMPACT} 

The RED-IMPACT assessment compared
results from European fuel cycle codes for various specific waste package
forms, radioactive and radiotoxic inventories, reprocessing discharges,  waste
package thermal power, corrosion of matrices, transport of radioisotopes and
resulting doses.  Granite, clay and salt were analyzed by various countries and 
codes as listed in table \ref{tab:red}.


%%%%%%%% RED-IMPACT Table %%%%%%%%%%%% 


\begin{table}
  \centering
  \footnotesize{
  \begin{tabular}{|l|c|c|c|c|r|}
    \multicolumn{6}{c}{\textbf{International Repository Concepts}}\\
    \hline
    Geology     & Nation      & Waste Stream   & Metric    & Institution & Code\\
    \hline 
    Granite     & Spain       & HLW            & Heat Load & Enresa & <+CodeName+> \\
    Granite     & Czech Rep.  & HLW            & Heat Load & NRI & <+CodeName+> \\
    Clay        & Belgium     & HLW            & Heat Load & SCK$\cdot$CEN & SAFIR2 \\
    Salt        & Germany     & HLW            & Heat Load & GRS & <+CodeName+> \\
    Granite     & Spain       & HLW            & Dose      & Enresa & <+CodeName+> \\
    Clay        & Belgium     & HLW            & Dose      & SCK$\cdot$CEN & SAFIR2 \\
    Clay        & France      & HLW            & Dose      & CEA & ANDRA \\
    Salt        & Germany     & HLW            & Dose      & GRS & <+CodeName+> \\
    Granite     & Czech Rep.  & ILW            & LT Dose   & NRI & <+CodeName+> \\
    Granite     & Spain       & ILW            & LT Dose   & Enresa & <+CodeName+> \\
    Clay        & Belgium     & ILW            & LT Dose   & SCK$\cdot$CEN & SAFIR2 \\
    Granite     & Spain       & HLW/ILW/Iodine & LT Dose   & Enresa & <+CodeName+> \\
    Clay        & Belgium     & HLW/ILW/Iodine & LT Dose   & SCK$\cdot$CEN & SAFIR2 \\
    \hline
  \end{tabular}
  \caption[International Repository Concepts]{International repository concepts evaluated in the RED Impact 
  Assessment.\cite{von_lensa_red-impact_2008}}
  \label{tab:red}
  }
\end{table}




%%%%%%%%%%%%%%%%%%%%%%%%%%%%%%%%%%%%%%

\clearpage

\subsection{Fuel Cycle Technology, UFD Codes Under Development}

The \gls{UFD} campaign is currently conducting an effort to produce
\gls{GDSE} models for various geological environments. Teams from
various national labs, Argonne, Lawrence Livermore, and Sandia are developing
models of generic clay, granite,   and salt disposal environments. Sandia is
simultaneously constructing  a borehole disposal system model. Each generic
disposal system model will perform detailed calculations of nuclide 
transport within its respective lithology. A longer term goal of the project 
includes an overarching generic model which incorporates the various 
geologically distinct submodels in order to provide a fully generic repository 
model. 

The nuclide transport calculations for the geologically distinct models 
are performed within the GoldSim simulation platform. GoldSim is a commercial
off the shelf simulation environment which ships with a contaminant transport 
system for modeling of release and transport of contaminants as well as an 
optional radionuclide transport module for decay calculations. Each models a 
single repeatable waste package cell to arrive a a full repository model. 
Probabilistic elements of the GoldSim modelling framework enable the models to 
incorporate \gls{FEPs} expected to take place probabilistically during the 
evolution of the repository.  

Cells within GoldSim represent components of the waste disposal system and
are linked by diffusive, advective, precipited, direct, or  otherwise filtered
mass transfer links. 

Thermal modeling for the \gls{GDSE} models are conducted independently with 
associated codes capable of modeling thermal evolution for all geologies. For 
example, one thermal model is being created using the SINDA$\\$G heat
transport solver and another is under development which utilizes a combination 
of  mathCAD and Excel. 

\subsubsection{Clay GDSE Model}

The Clay \gls{GDSE} model is being pursued by the team at \gls{ANL} and will be 
the primary model with which this work will conduct parametric regression 
analyses. 

The Clay \gls{GDSE} models a single waste form, waste package, \gls{EBS}, 
\gls{EDZ}, and far field zone.

<++>

\subsubsection{Salt GDSE Model}

<++>


\subsubsection{Granite GDSE Model}

<++>

\subsubsection{ Borehole GDSE Model}


<++>


%%%%%%%%%%%%%%%%%%%%%%%%%%%%%%%%%%%%%% 
\section{Analytical Models of Nuclide Transport} \label{sec:analytical_nuc}

%%%%%%%%%%%%%%%% Analytical Nuclide Transport %%%%%%%%%%%%%%%%%%%%%%%%

%%%%%%%%%%%%%%%%%%%%%%%%%%%%%%%%%%%%%% 
%% Source Term in Many Geos Table %%%%
%%%%%%%%%%%%%%%%%%%%%%%%%%%%%%%%%%%%%% 

  \begin{table}
    \centering
    \footnotesize{
    \begin{tabular}{|l|c|c|l|}
      \multicolumn{4}{c}{\textbf{Models of Source Term for Various Geologies}}\\
      \hline
      Source & Nation & Geology & Methodology \\  
      (Who) & (Where) & (What) & (How) \\  
      \hline
      Enresa \cite{von_lensa_red-impact_2008}           & Spain       & Granite                   &  GoldSim \\ 
                                                        &             &                           & $^{129}I$ primary contributor \\
      SCK$\cdot$CEN   \cite{von_lensa_red-impact_2008}  & Belgium     & Clay                      & FEP\\
                                                        &             &                           & $^{129}I$ primary contributor \\
      GRS \cite{von_lensa_red-impact_2008}              & Germany     & Salt                      & Systematic Performance Asessment \\
                                                        &             &                           & $^{135}$Cs, $^{129}$I, $^{226}$Ra, $^{229}$Th \\
      Ahn \cite{ahn_environmental_2004, ahn_environmental_2007} & USA     & Yucca Tuff                & Solubility Limited Release \& \\ 
                                                        &             &                           & Congruent Release  \\
      NCSU(Nicholson) \cite{li_methodology_2006}        & USA         & Yucca Tuff                & TSPA codes EBSREL and EBSFAIL  \\ 
      WIPP                                              & USA         & Salt                      & ?  \\
      NAGRA \cite{johnson_project_2002, johnson_calculations_2002}  & Switzerland & Opalinus Clay & TAME code  \\
      ANDRA \cite{andra_argile:_2005}                   & France      & Argile Clay               & Very detailed CEA code  \\
                                                        &             &                           & Mostly homogeneous medium \\
                                                        &             &                           & $^{129}I$ primary contributor \\
      ANDRA \cite{andra_granite:_2005}                  & France      & Granite                   &  Very detailed CEA code  \\
                                                        &             &                           &  Involves fracturation of medium \\
                                                        &             &                           & $^{129}I$ primary contributor \\
      SKB \cite{ab_long-term_2006}                      & Sweden      & Forsmark                  &  HYDRASTAR  \\
                                                        &             & Laxemar                   &  hydrologic transport code\\
      \hline
    \end{tabular}
    \caption[Models of Source Term for Various Geologies]{Methods by which to evaluate source term dependence of waste package failure, transport through the EBS and hydrogeologic transport. The latter two parts vary significantly among host formations. }
    \label{tab:geosource}
    }
  \end{table}


%%%%%%%%%%%%%%%%%%%%%%%%%%%%%%%%%%%%%%

A comprehensive model of radiotoxic source term must address nuclide transport
through the full release pathway including waste packages, engineered barrier
systems, and geologic media. A model of transport through the waste package
must incorporate waste package failure rate, nuclide release rate via waste
form dissolution, and advective transfer rate into the engineered barrier
system.  

Waste package failure rate depends on near field environmental factors such as
pH and humidity as well as decay heat and radiative damage anticipated from the
contained waste.  In turn, the nuclide release rate from the waste package
depends on the character of the waste form matrix, treatment of water flow,
nuclide solubility and the elemental diffusion constant.  Similarly, advective
transfer through the engineered barrier system and into the geological medium
also depends on water flow, nuclide solubility, and nuclide diffusion, but must
be employed in the context of the hydrogeology of the rock.   

Waste package failure rate varies between models. While some employ detailed 
computational tools such as GoldSim or EBSFAIL (a part of the EBSPAC module 
used in the TSPA code) which will be discussed in section 
\ref{sec:detailed_nuclide}, 
analytic models incorporate their own hydrogeologic approximations of
canister degredation or make simpler assumptions of immediate waste canister 
failure in order to focus on dissolution and transfer. 

Waste package release rate is the rate of mass transfer of a nuclide from its
waste form into the saturation water. The mode of water flowthrough heavily
effects nuclide dissolution rate and is treated differently in various models.
While some, inspired by the TSP assessment, assume water moves through the
waste packages at a constant volumetric rate (`flowthrough model'), others
adopt less conservative assumptions incorporating weather based predictions of
hydrogeologic activity.

Nuclide transfer rate through the lithology is  dependent upon diffusion as
well as advection.  The diffusion coefficient varies per nuclide and is heavily
dependent upon the concentration of that nuclide in the flowthrough water.
The way in which diffusion affects nuclide concentration is described by Fick's 
Second Law, 

\begin{equation} 
  \frac{\partial C_i}{\partial t}  = D_i\nabla^2 C_i.  
\end{equation}

Source term dependence on concentration has a significant effect on potential
repository capacity. Sensitivity to concentration complicates the viability of
alternate loading schemes as well as waste separation scenarios. Specifically, 
Fick's Second Law becomes 

\begin{equation} 
  \frac{\partial C_i}{\partial t}  = \nabla \cdot ( D_i \nabla C_i ). 
\end{equation}

Radionuclide concentration has been shown to be proportional to the waste package
loading configuration for the Yucca Mountain case. 
\cite{ahn_relationship_2002,kawasaki_congruent_2004}

The most important geochemical processes to source term that occur within the
waste, waste matrix, and near field as well as the host rock are dissolution,
precipitation, and sorption.  \cite{bracke_safety_2008}

\subsubsection{Dissolution}

Dissolution is the ``most important prerequisite'' for contribution of a
radionuclide to source term.\cite{bracke_safety_2008} That is, the initial 
dissolution of a radionuclide from its waste from is a breach of the ultimate 
barrier, or final line of defense. Dissolution rates depend strongly on pH, Eh,
speciation of radionuclides (chemical oxidative state), and radiolysis of the
dissolving fluid.

\subsubsection{Precipitation}

Precipitation is the reverse of dissolution and occurs when a soubility limit
is reached. Interestingly the chemical equillibrium of dissolution and
precipitation is rate dependent and is highly dependent on thermodynamics.
Predictions are not reliable in may cases.

\subsubsection{Sorption}

Sorption is the interaction of a dissolved species with surfaces that removes
that species from the dissolving medium models.


Sorption into the rock matrix is a method by which contaminants (and water? ) %
are removed from a fracture during flow through that fracture. However, this is 
a reversible process \cite{ahn_mass_1988}, which means the contaminant 
might be returned to the fracture from the matrix with the same ''distribution 
coefficient`` with which they entered the rock matrix . 

% One way to model sorption is by assuming a linear isotherm. <+What does that 
% mean? +> \cite{ahn_thesis_pg20}

% The concentration of sorbed contaminant onto the fracture surface can be 
% demonstrated in terms of a rate equation (if surface diffusion is neglected.) 
% (What is surface diffusion?) \cite{ahn_thesis_pg20)


% Ahn finds that including sorption, the concentration of contaminant in the 
% fracture satisfies the differential rate equation: % \begin{align*} 
% R_f\frac{\partial N}{\partial t} + \nu \frac{\partial N}{\partial z} 
% - D \frac{\partial^2 D}{\partial z^2} + R_f \lambda N + \frac{q}{b} = 0, z>0 
%   and t>0 
% \end{align*} 

% Also, there is sorption through the pores in the matrix. Again, we can model 
% this with a ``linear sorption isotherm. . . '' whatever that is.  



\subsubsection{Colloidal Mobility}

Colloids present in the near field dissolving solution have an effect on the
mobility of radionuclides. Studies addressing the subtle differences between 
resultant behavior of various isotopes exist.

\subsection{Waste Form Release Models}

\subsubsection{Hedin Model (\cite{hedin_integrated_2002})}

In a saturated fractured rock matrix representative of the KBS-3 granitic
Swedish repository concept, copper canister waste packages contain a waste
matrix, and a bentonite buffer surrounds the canisters within repository drift
tunnels. Waste form dissolution within the Hedin model is a rate based model
which takes place within the waste package void. Nuclides are released
congruently until the waste form is completely degraded.
\cite{hedin_integrated_2002} 

\subsubsection{Ahn Models (\cite{ahn_environmental_2004,
ahn_environmental_2007})}

Waste canisters are modelled as compartments of waste form surrounded by a
buffer layer which is in turn surrounded by layers of near field rock and far
field rock. Water is introduced to the system at a constant rate, and
encounters an array of failed waste packages (at $t=0$ in the 2004 model, and
at $T_f=75,000$ years in the 2007 model). The water immediately begins
dissolving the waste matrix.  Nuclides with higher solubilities are
preferentially dissolved and treated with a `congruent release' model discussed
below. Nuclides with lower solubilities are transported through the buffer with
the alternative `solubility limited' release model. The water flow begins at
one waste package and travels through the matrix and buffer space to the next
waste package, contacting each waste package consecutively and then flowing on
into the near field. In this way, the water is increasingly contaminated as its
path through the waste packages proceeds.  

\subsubsection{Congruent Release Model} 

In the Ahn models, nuclides with a high solubility coefficient are modeled with
the congruent release model.  Nuclides of this type include most of the fission
products, but not the actinides. This model states that the release from the
waste packages is congruent with the dissolution of the waste matrix and is
transported through the rock by advective transfer with the water that flows
through the waste packages.  

\subsubsection{Solubility Limited Release Model}

In the Ahn models, nuclides with lower solubility coefficients are modeled with
the solubility limited release model.  Solubility values are assumed from TSPA
for this model, and a solubility of $~5\times 1^{-2} [mol/m^3]$ are taken to be
`low.' Elements in this `low' category include the toxic actinides such as Zr,
Nb, Sn, Th, and Ra.  This model suggests that a dominant mode of dissolution of
the nuclide into the flowthrough water is dominated instead by the diffusion
coefficient, which is largely dependent upon the concentration gradient between
the waste matrix and the water. The mass balance driving nuclide release takes
the form:

\begin{equation} \dot{m_i}=8\epsilon D_eS_iL\sqrt{\frac{Ur_0}{\pi D_e}}
\end{equation} where $\epsilon$, U, $r_0$, and L are the geometric and
hydrogeologic factors porosity, water velocity, waste package radius, and waste
package length, repsectively. $D_e$ is the diffusion coefficient ($m^2/yr$) of
the element \emph{e} and $S_i$ is the isotope's solubility ($kg/m^3$).

In the Hedin model of the waste matrix, the amount of solute available within
the waste package is solved for, and for nuclides with low solubility, the mass
fraction released from the waste matrix is limited by a simplified description
of their solubility. That is, 

\begin{align*} m_{1i}(t)\le v_{1i}(t)C_{sol} \end{align*}

where the mass $m_{1i}$ $[g]$ of a nuclide, $i$ relased into the waste package
void volume $v_1$ in $[m^3]$, at a time t, is limited by constant the maximum
concentration, $C_{sol}$ in $[g/m^3]$ at which that nuclide is soluble.
\cite{hedin_integrated_2002}

%%%%

\subsection{Waste Package Failure Models}

%%%%%%%%%%%%%%%%%%%%%%%%%%%%%%%%%%%%%% 
%%%%% WP Failure Modes Table %%%%%%%%%
%%%%%%%%%%%%%%%%%%%%%%%%%%%%%%%%%%%%%% 
\begin{table}
\centering
\footnotesize{
\begin{tabular}[h!bt]{|l|r|r|r|}
  \multicolumn{4}{c}{\textbf{Current Waste Package Failure Models}}\\
  \hline
  Model&WP Failure Mode&Waste Form&Details\\
  \hline
  TSPA&EBSFAIL&&$300,000$ years\\
  \hline
  Ahn 2003&Instantaneous Failure&Borosilicate Glass&$t=0$\\
  \hline
  Ahn 2007& &CSNF $UO_2$ matrix &$T_f=75,000$ years\\
  & &Borosilicate Glass &$T_f=75,000$ years\\
  & & Naval $UO_2$ matrix &$T_f=75,000$ years\\
  \hline
  Li&EBSFAIL&&$300,000$ years\\
  \hline
  Hedin 2003& Instantaneous & Copper KBS-3 Concept & $t_{delay} = 300$ years \\
  \hline
\end{tabular}
\label{tab:wpfail}
\caption[Current WP Failure Models]{The above represent current methods by which waste packeage 
failure rates are modeled.}
}
\end{table}

%%%%%%%%%%%%%%%%%%%%%%%%%%%%%%%%%%%%%%


\subsubsection{Instantaneous}
 
The Hedin model of waste package failure is effectively instantaneous, but
limited by a release resistance coefficient. The release is assumed  to occur
through a hole in the waste canister that exists throughout the simulation, and
the resistance coefficient limiting flow through the hole represents the
magnitude of the canister flaw in combination with the buffer-geosphere
interface.  \cite{hedin_integrated_2002}

Other models also use an instantaneous waste package failure mode for all waste  
packages simultaneously, assuming that either at the beginning of the simulation
or at the onset of the simulation. The mathematical representation of this 
probability density function is clearly just the delta function, with $n_F$ 
being the number of failed waste packages, $N$ being the total number of waste 
packages, and $t_F$ being the time to failure.
\begin{align}
  n_F(t) = N\delta(t-t_F)
  \label{instantaneous}
\end{align}

\subsubsection{Rate Based} 

When enough data exists, waste package failure can
be represented more  realistically by fractional destruction according to
experimentally observed corrosion and dissolution rate functions.

However, this can be complicated to model even if the data exists. In
particular, the corrosion rate will depend dramatically on hydrologic and
thermal conditions. Specificially, corrosion rates for the same material are
very different under dry oxidising conditions and wet reducing conditions. 

%almost no corrosion aqueous oxidising conditions, which means a steel corrosion 
%rate (concentrated at the cell head) of several tens of microns per year. This 
%phase is very limited in
%duration; or under anoxic and reducing conditions.  during which the steel 
%corrosion rate is reduced to a few microns (or even less than a micron) per 
%year.  see pg 283/525 of phenomenological evolution doc for details

An expression of the number of failed waste packages per year is a simple 
functional relationship. 

\begin{align}
  n_F = Nf(t,T,\cdots)
  \label{rate}
\end{align}

\subsubsection{Probabilistic}

When a probability distribution of waste package failure is available, the 
discrete waste packages can be modeled to fail according to that distribution. 
For example, if the expected lifetime of a waste package is 30,000 years, a 
gaussian distribution around that number would provide a probabilistic number of 
waste package failures per time step. 

\begin{align}
  n_F=Nf(t)
  \label{probabilistic}
\end{align}

The instantaneous case is a special case of the probabilistic situation. 
Specfically, it is the Dirac delta function mentioned in equation 
\eqref{intantaneous}.

A particularly attractive probability distribution for use in the case of failed 
engineered barriers is the Weibull distribution. <++>

%%%%

\subsection{Nuclide Transport Through Secondary Engineered Barriers}

When the waste package is breached, nuclides are transported through secondary 
engineered barriers which includes the buffer zone and tunnel walls. After 
transportation through the secondary EBS nuclides reach the geological matrix. 

\subsubsection{Barrier Dissolution and Failure}

The same models of waste package failure (instantaneous, rate based, and 
probabilistic) can be applied to buffers.

\subsubsection{Transport Through Degraded Barrier Matrix}

Diffusive and advective transport occur in the barrier matrix both before and 
after degradation. Before degradation, transport is primarily diffusive. 
Thereafter, transport is primarily advective. Cracking can be modeled explicitly 
or as a continuum model.  

%%%%

\subsection{Hydrogeologic Transport Models}
Clay, granite, and salt can largely be characterized as permeable porous media. 
Solute transport in permeable porous media involves advection, hydraulic 
dispersion, and diffusion. 

Advection is transport congruent with the water velocity, diffusion is the 
result of brownian motion accross concentration gradients, and hydraulic 
dispersion is transport resulting from anisotropies in the water velocity field. 



% Solute Transport in permeable porous media 

% advection - transport at the velocity of water flow 

% hydraulic dispersion - transport due to anisotropies in the water 

% velocity feild. It depends on concentration of solute, darcy 

% velocity, and dispersivity.  

% diffusion - is from random brownian motion, tends to homogenize 
% concentration field.  %

The equation representing solute transport in a permeable medium of homogenous
porosity can be written (as in Schwartz and Zhang 
\cite{schwartz_fundamentals_2003})

\begin{align} 
  \frac{\partial \omega C}{\partial t} & = - \nabla \cdot  (F_c + F_{dc} + F_d) + m 
  \label{solperm}
  \intertext{where} 
  \omega &= \mbox{solute accessible porosity } [\%]\nonumber\\
  C &= \mbox{ concentration } [kg \cdot m^{-3}]\nonumber\\ 
  F_c &= \mbox{ convetive flow } [kg \cdot m^{-2}\cdot s^{-1}]\nonumber\\
  &= qC \nonumber \\
  F_{dc} &= \mbox{ dispersive flow } [kg \cdot m^{-2}\cdot s^{-1}]\nonumber\\ 
  &= \alpha q \nabla C  \nonumber\\ 
  F_d &= \mbox{ diffusive flow } [kg \cdot m^{-2}\cdot s^{-1}]\nonumber\\
  &= D_e \nabla C\nonumber\\
  m &= \mbox{ solute source } [kg \cdot m^{-2}\cdot s^{-1}].\nonumber
\end{align} 

In the expressions above, 

\begin{align*} 
  q &= \mbox{ Darcy velocity } [m\cdot s^{-1}] \\ 
  \alpha &= \mbox{ dispersivity } [m]
  \intertext{and} D_e &= \mbox{ effective diffusion coefficient } [m^2\cdot s^{-1}] .\\ 
\end{align*} 

The method by which the dominant solute transport mode (diffusive or advective)
is determined for a particular porous medium is by use of the dimensionless
Peclet number, 

\begin{align*} 
  Pe &= \frac{qL}{\alpha q + D_e},\\
  &= \frac{\mbox{advective rate}}{\mbox{diffusive rate}}
  \intertext{where} 
  L &= \mbox{ transport distance } [m].\\ 
\end{align*}

For a high $Pe$ number, advection is the dominant transport mode, while 
diffusive transport dominates for a low $Pe$ number.  

The analytical expression in equation \eqref{solperm} will be the foundation of 
simplification by regression analyses for the nuclide transport interface 
between components of the repository system model representing permeable porous 
media.  

\subsubsection{Solute Transport in Fractured Media} 

\paragraph{Continuum Models} 

The models arrived at via this genre of approximation are appropriate
for very fractured or very unfractured situations.

Equivalent Porous Medium (EPM) models assert that a uniformly  fractured medium
can be approximated as a fractureless matrix with an effective porosity high
enough to account for real fracturation.  \cite{berkowitz_continuum_1988}
\cite{anderson_applied_1992}


Dual Porosity Models make up one type. This model incorporates advective
transport in simplistic, uniform fractures and diffisive sorption and
desorption into the stagnant (no advective transfer) water contained in the
pores of the rock matrix \cite{uleberg_dual_1996} \cite{ho_dual_2000}.


Dual Permeability Models are another type. These are similar to dual porosity
models, but incorporate advective transfer within the rock matrix and between
the rock matrix and the fracture volume.  \cite{uleberg_dual_1996}
\cite{ho_dual_2000}

\paragraph{Discrete Fracture Network Models} Discrete fracture network models
approximate that water and contaminants move only through the fracture network
\cite{anderson_applied_1992} \cite{schwartz_fundamentals_2003}.

The flow in each fracture can be approximated, as in Schwartz and Zhang, with
the flow between two parallel plates having an aperture $b$, the mean fracture
height \cite{schwartz_fundamentals_2003}. For a fracture perpendicular to
gravitational acceleration, $g$, the hydraulic conductivity, $K$, is described
according to the cubic law as 

\begin{align} 
  K&= \frac{\rho_w g b^2}{12 \mu} \label{Kplates} 
  \intertext{where}
  \rho_w &= \mbox{water density,}\nonumber\\ 
  \mu &= \mbox{viscosity}.\nonumber
\end{align}

Accordingly, the volumetric flow rate in the single fracture of width, $w$, can
be described in terms of the hydraulic head gradient, $\frac{\partial
h}{\partial L}$, as

\begin{align} Q & = -Kbw\frac{\partial h}{\partial L} \label{Qplates}
\end{align}

Calculation of the volumetric flow rate and corresponding solute transport in a
discrete fracture network model for many non-parallel fractures is an intensive
numerical computation. However, for uniformly fractured media, a fracture
network can be approximated by a set of parallel plates fractures. 

If flow is expect in the $\theta_f$ direction, and the fractures of the set are
spaced a distance, $s$, apart,

\begin{align} 
  N &= \mbox{fracture frequency}\nonumber\\ 
  &= \frac{cos(\theta_f)}{s}.  
  \label{fracfreq} 
\end{align}

The fracture network permeability is then defined as, 

\begin{align} 
  k_f =
\frac{b^3}{12N}.  
\label{fracperm} 
\end{align}

The permeability, $k$, in an equivalent permeability model is thereby obtained
by the permeabilities of the fracture network, $k_f$, and the permeability of
the host matrix, $k_m$. Following the derivation in Schwartz and Zhang, in
terms of the cross-sectional contact areas of the matrix and fractures $A_m$
and $A_f$, the equivalent permeability, $k$, can be expressed

\begin{align} k = \frac{k_m + \frac{A_f}{A_m}k_f}{1+\frac{A_f}{A_m}}.
\label{equivperm} \end{align}

\subsection{Geochemical Transport Models}

\subsubsection{Reducing Environments}

<++>

%\paragraph{Saturated Environments}

%\paragraph{Unsaturated Environments}

\subsubsection{Oxidizing Environments}

<++>

%\paragraph{Saturated Environments}

%\paragraph{Unsaturated Environments}


% The distribution of fracture aperture sizes in a saturated fractured hard rock 
% matrix determines the appropriate model with which to treat rock fracturation 
% and subsequent nuclide transport through fracturous pathways. 

% Fracturation within a saturated medium can be treated either as an effective 
% porosity, in which n_eff is suggested according to the quantity of connected 
% fractures, much like connected pores, in a rock matrix. If these are of 
% similar sizes, 

% Fracturation in grantite, tuff, and basalt have a log normal distribution of 
% of fracture apertures. Therefore, there are a few large fractures but the 
% majority are microfractures.  Therefore, the equivalent porous medium approach 
% is inappropriate. Instead we must take the approach in which fractures of 
% larger aperture are given greater importance and are modeled as primary 
% conduits to the biosphere while minor fractures can be considered a part of 
% the porous medium of the saturated rock. \cite{ahn_thesis}


% For details about fracturation models in unsaturated rock, it's likely best to 
% consult YMR TSPA models.


%\paragraph{Diffusion Into Fracture}

%\paragraph{Advection Through Fracture}

%\paragraph{Effective Porosity}

%\paragraph{Major and Minor Fracturation}



\section{Analytical Models of Heat Transport} \label{sec:analytical_heat}

%%%%%%%%%%%%%%%% Analytical Heat Transport %%%%%%%%%%%%%%%%%%%%%%%%
\subsection{Conduction}

Conductive heat transfer occurs as a result of a temperature gradient. Heat 
flows diffusively from the hotter material to the cooler material over time and
steadily approaches thermal equillibrium. A heat flux, $\vec{q} [W\cdot 
m^{-2}]$ can be expressed with the  differential form of Fourier's law of heat 
conduction with respect to  the thermal conductivity of the material, $k [W 
\cdot m^{-1} \cdot ^\circ K^{-1}]$, and the temperature gradient $\nabla T [K/m]$

\begin{align}
  \vec{q}= -k\nabla T.
  \label{fourier}
\end{align}

For the one dimensional case, equation \ref{fourier} can be reduce using a 
finite difference approximation. For a body at $x_1$ with temperature $T_1$
and a body with temperature $T_2$ at position $x_2$,

\begin{align*}
  q_x &= \frac{dT}{dx}\\
  &=-k_x\frac{(T_1-T_2)}{x_1-x_2}.
\end{align*}


\subsection{Convection}

Convective heat transfer occurs advectively in accordance with fluid movement. 
The form of Fourier's Law which describes convection is very similar to equation  
\ref{fourier} for conduction, though the convective heat transfer coefficient 
$h [W m^{-2} K^{-1}]$ and the surface area of heat transfer $A [m^2]$ replace 
the conductivity coefficient. Convection is typically expressed as

\begin{align}
  \vec{q}&= -hA\nabla T
  \intertext{such that}
  q_x &= hA\frac{dT}{dx}\\
  &=-h_xA\frac{(T_1-T_2)}{x_1-x_2}.
\end{align}


\subsection{Radiation}

Heat transfer by radiation is the result of the emmission of electromagnetic 
waves. Plack black body radiation is analytically described, using $\sigma$, the   
Stefan-Boltzmann constant as

\begin{align}
  \vec{q} &= \sigma A_1 F_{1\rightarrow 2}(T_1^4 - T_2^4)
  \label{planck}
  \intertext{where}
  \sigma &=5.670373\times 10^{−8} [W\cdot m^{−2}\cdot ^\circ K^{−4}].\nonumber\\
\end{align}


\subsection{Mass Transfer}

Heat transfer by mass transfer is straightforward, resulting in the change in 
temperature in adjacent volumes as a result of matter movement. If the specific 
heat capacity of the transferred mass can be expressed as $c_p$, then the heat 
conductance is simply, 

\begin{align*}
  G = \dot{m}c_p.\\
\end{align*}

\subsection{Lumped Parameter Model}

The lumped heat capacitance model makes an analogy to electrical circuit by 
reducing a thermal system into discrete lumps. Such an approximation is 
appropriate when it can be assumed that the temperature gradient within each 
lump is approximately uniform. the appropriateness of this approximation can 
quantitatively expressed by comparison of the internal thermal resistance to the 
external thermal resistance. The Biot number, 

\begin{align}
  Bi = \frac{hA}{k}\\
  \label{biot}
\end{align}

indicates the relative speeds with which heat conducts within an object and 
accross the boundary of that object. If the Biot number is low $(<0.1)$, and 
therefore conduction is faster within the object than at the boundary, the 
assumption of a uniform internal temperature is appropriate and the lumped 
parameter model may be expected to give a result within $5\%$ 
error.\cite{incropera_fundamentals_2006} 

The lumped capacitance model can address multiple media and multiple heat
transfer modes. The rate of heat transfer $[W \cdot m^{-2} \cdot ^\circ K \cdot 
s]$ through a circuit is simply given as the quotient of the temperature 
difference and the thermal resistance of the multiple lumps,

\begin{align*}
  \dot{q} = \frac{\Delta T}{\sum _{i=0}^{N}R_i}.
\end{align*}

By representing the various modes of heat transport (i.e. conduction, 
convection, radiation, and mass transfer) with various expressions for 
resistance, the lumped capacitance model provides a solution to the transient 
problem described by the energy balance,

\begin{align*}
  \left( \mbox{Energy added to body j in dt} \right) &= \left( \mbox{Heat 
  out of adjacent bodies into body j} \right)\\
  c_j\rho_j V_j dT_j(t) &= \sum_{i=0}^{i=N}\left[\dot{q_{i,j}}\right]dt,
\end{align*}

where $c_j\rho_jV_j$ is the total lumped thermal capacitance of the body.

For the case of a simple conductive circuit between two bodies, $i$ and $j$, the 
resistance of $j$ can be described as 
\begin{align*}
  R_{cond} &= 1/hA
  \intertext{such that}
  c_j\rho_j V_j dT_j(t) &= \sum_{i=0}^{i=N}\left[ hA_j [T_i - T_j(t) \right]dt\\
\end{align*}

Under integration a time constant appears which describes the speed with which 
the body $i$ changes temperature with respect to the maximum temperature change,

\begin{align*}
  \int_{T_j=T_0}^{T_j(t)} 
  \frac{dT_j(t)}{T_i-T_j}&=\frac{hA_j}{c_j\rho_jV_j}\int_0^t dt\\
  -ln\frac{T_i-T_j(t)}{T_i-T_0}&=\frac{hA_j}{c_j\rho_jV_j}t\\
  \frac{T_i-T_j(t)}{T_i-T_0}&= e^{-(hA_j/c_j\rho_jV_j)t}\\
  \intertext{such that}
  \frac{T_j(t)-T_i}{T_i-T_0}&= 1- e^{-t/\tau}\\
  \intertext{where}
  \tau &= \left( c_j\rho_jV_j/hA_j \right).
\end{align*}

The time constant, $\tau$ is the time it takes for the body to change 
$(1-(1/e))\%\Delta T$ and is equal to the product of the thermal capacitance and 
thermal resistance of the body, $CR$, analgous to an electrical circuit.
\cite{el-wakil_nuclear_1981} This is the case for all resistances, $R_i$ 
representing modes of heat transfer. Thus, one can say, in general

\begin{align*}
  \tau_j = c_j \rho_j V_j R_j.
\end{align*}

\subsection{Impact of Repository Designs}

\subsection{Heat Limits in Various Waste Packages} 
% The CEA and ANDRA take 90$^\circ$C to be the maximum temperature for spent 
% fuel waste packages. The reason for this is to remain within well understood 
% limits of material evolution of waste packages and the surrounding (clay) 
% repository.  \cite{argile_geo_evo}  


% Confinement capabilities of bitumen waste packages (B packages in ANDRA Argile 
% evaluation) is only confident below 70$^\circ$C. Uncide the ``disposal cells'' 
% a maximum temperature of 30$\circ$C is adopted.


\subsection{Heat Limits in Various Geologies}

%%%%%%%%%%%%%%%%%%%%%%%%%%%%%%%%%%%%%%%%%%%%%%%% 
%%%%% Heat Load Computational Models Table %%%%% 
%%%%%%%%%%%%%%%%%%%%%%%%%%%%%%%%%%%%%%%%%%%%%%%% 

 \begin{table}[h!]
    \centering
    \footnotesize{
    \begin{tabular}{|l|c|c|l|}
      \multicolumn{4}{c}{\textbf{Models of Heat Load for Various Geologies}}\\
      \hline
      Source & Nation & Geology & Methodology \\  
      (Who) & (Where) & (What) & (How) \\  
      \hline
      Enresa \cite{von_lensa_red-impact_2008}           & Spain       & Granite       &  CODE\_BRIGHT  \\ 
      NRI   \cite{von_lensa_red-impact_2008}            & Czech Rep.  & Granite       &  Specific Temperature Integral   \\
      ANDRA \cite{andra_granite:_2005}                  & France      & Granite       &  3D Finite Element CGM code   \\
      SKB \cite{ab_long-term_2006}                      & Sweden      & metagranite   &  Forsmark / Laxemar Site \\
                                                        &             &               &  Descriptive Model (SDM)\\
      SCK$\cdot$CEN   \cite{von_lensa_red-impact_2008}  & Belgium     & Clay          &  Specific Temperature Integral   \\ 
      ANDRA \cite{andra_argile:_2005}                   & France      & Argile Clay   &  3D Finite Element CGM code   \\
      NAGRA \cite{johnson_project_2002, johnson_calculations_2002}  & Switzerland  & Opalinus Clay &  3D Finite Element CGM code \\
      GRS \cite{von_lensa_red-impact_2008}              & Germany     & Salt          &  HEATING (3D finite difference)   \\ 
      NCSU(Li)   \cite{li_examining_2007}               & USA         & Yucca Tuff    &  Specific Temperature Integral \\        
      NCSU(Nicholson) \cite{nicholson_thermal_2007}     & USA         & Yucca Tuff    &  SRTA and COSMOL codes\\
      Radel \& Wilson \cite{radel_repository_2007}      & USA         & Yucca Tuff    &  Specific Temperature Change \\ 
      \hline
    \end{tabular}
    \caption[Models for Heat Transport for Various Geologies]{Methods by which to calculate heat 
    load are independent of geology. Maximum heat load constraints, however, vary among host formations. }
    \label{tab:heat}
    }
  \end{table}


%%%%%%%%%%%%%%%%%%%%%%%%%%%%%%%%%%%%%%

The repository concepts under consideration are closed concepts. That is, 
closure of the repository happens very shortly after emplacement.
Enclosed modes are appropriate for low permeability rock formations (clay/shale, 
granite, salt). Low permeability does not permit oxygen entry, so these are
reducing environments. Since heat is carried away less effectively once closure 
has occured, heat limits are lower in  closed repository concepts than in open 
ones such as \gls{YMR} where drift tunnels are ventilated for a number of years
after emplacement before closure.

\subsubsection{Clay} 

Heat limits in clay are based on the domain of known behavior in clay and the 
tendency for bentonite fill material to lose its isolating properties with heat. 
The alteration high smectite bentonite to non-expandable clays is a primary 
limitation for heat tolerance in the clay concept. The isolation characteristics 
of bentonite buffer materials is reduced after this alteration. The time 
integral of this phenomenon determines total bentonite alteration. While short 
bursts of heat might be allowable, because the bentonite will not alter 
immediately, the kinetic alteration into smectite clays is hastened by 
temperatures above approximately 100$^\circ C$. \cite{pusch_alteration_1987} 
Well understood behavior for agrillaceous clay and bentonite buffer backfill
is conservatively assumed by the \gls{ANDRA} assessment to occur under 
90$^\circ C$, which is effectively a limit at the waste package interface with 
the bentonite buffer material.\cite{andra_argile:_2005} 
The \gls{NAGRA} Opalinus Clay assessment less conservatively 
uses a maximum heat limit in the bentonite buffer of $125^\circ C$.
\cite{johnson_project_2002} 

%as with the waste packages in this geology.  

% 
% ``This low permeability of the Callovo-Oxfordian, linked to low hydraulic head 
% gradients on either side of the formation, controls slow vertical water flow 
% (Inset 3.5). The velocities of this water flow are fairly different in detail 
% because of the medium's pore structure. Moreover, some pores are not connected 
% and cannot take part in the water flow. A so-called kinematic porosity is 
% therefore defined, which is a fraction of total porosity, used macroscopically 
% to calculate mean water flow velocity in the direction of the hydraulic head 
% gradient. For the Callovo-Oxfordian, this kinematic porosity has been taken as 
% being the same as the fraction of free water in the rock, i.e. about 9 
%, corresponding to half the total porosity.  The very low permeability 
%determines average flow speeds within the layer (Darcy velocity, inset 3.5) at 
%around 3 cm per 100,000 years, which corresponds to a water transfer velocity 
%of about 30 cm per 100,000 years, considering the kinematic porosity.  
%\cite{argile_geo_evo} 

% Water flow in a porous medium is dominated by darcy flow.  This would be a 
% great place to describe darcy's law! 

\subsubsection{Granite}

Granite repository concepts are limited by the bentonite buffer in a manner 
similar to that of clay. 
Well understood behavior for generic granite and bentonite buffer backfill
is conservatively assumed by the \gls{ANDRA} assessment to occur under 
90$^\circ C$, a limit that must be imposed at the waste package interface with 
the buffer material.  Mechanical stresses and strains due to heating at this 
level were analyzed by \gls{ANDRA} and shown to have a negligible effect of 
flow behavior. Similarly, thermo-hydraulic effects due to thermally induced 
fluid density changes are expected to be slight. \cite{andra_argile:_2005}
Similarly, for reasons of buffer isolation integrity, the Czech and Spanish
granite disposal concepts both maintained a thermal limit at the waste package
interface with the buffer of $100^\circ C$.  \cite{von_lensa_red-impact_2008}

\subsubsection{Salt} 

Response of a salt repository to heat has a significant
mechanical component. Bulk heating of a salt repository matrix causes
coalescing  of the salt surrounding the heat source. In the case of a nuclear
waste repository, this phenomenon increases isolation capability of the salt. A
heat limit, then, is difficult to characterize, but evolution of the heat in a
salt environment is of great importance to nuclide transport modeling. 

A model of temperature dependent salt coalescent behavior is is order. 

The \gls{UFD} salt repository concept was based on some 
experience trying to construct and maintain the horizontal borings at WIPP. The 
current geometry involves emplacement of crushed salt in an alcove over waste
packages which are arranged at the corner of the alcove. Notably,  crushed salt 
has low conductivity, which increases the sensitivity of rock salt temperature 
on emplaced package temperature. 

There is a current open question involving the behavior of moisture under high 
heat. Though it is clear that the salt will creep and coalesce with higher heat, 
the movement of brine within the salt at high heat is a pertinent issue for the 
understanding of nuclide transport within salt.

The German salt repository concept maintains a $180^\circ C$ temperature limit 
in the salt. This requirement is based on the geochemical stability of the rock  
salt. At temperatures above $220^\circ C$ the salt formation may release 
brines.\cite{von_lensa_red-impact_2008}\cite{brewitz_long_2002}


\subsubsection{Unsaturated Tuff}

For comparision with the geologic media under consideration, it is important to 
note that two heat load constraints primarily determine the heat-based SNF capacity
in the Yucca Mountain Repository design, which is located in unsaturated tuff.
Thermal limits in that design are intended to passively steward the
repository's integrity against radionuclide release for the upcoming 10,000
years.

The first constraint intends to prevent repository flooding and subsequent
contaminated water flow through the repository. It requires that the minimum
temperature in the granite tuff between drifts be no more than the boiling
temperature of water which, at the altitude in question, is $96^{\circ}C$. This
is effectively a limit on the temperature halfway between adjacent drifts,
where the temperature will be at a minimum.

The second constraint intends to prevent high rock temperatures that induce
fractures and would increase leach rates. It states that no part of the rock
reach a temperature above $200^{\circ}C$, and is effectively a limit on the
temperature at the drift wall, where the rock temperature is a maximum.  

The statutory limit of once-through, thermal PWR waste is 70,000 tonnes SNF.
That is to say, the statutory line load limit is approximately 1.04 tonnes/m
for 67km of planned emplacement tunnels (with 81 meters between drifts). The
Office of Civilian Radioactive Waste Management Science and Engineering Report
gives this basic ``statutory limit", but suggests an inherent design
flexibility that could allow for expansion. The ``full inventory" Yucca
Mountain design alternative gives a maximum repository capacity of 97,000
tonnes. In addition, the current design for the repository has flexibility for
``additional repository capacity" which would give a 119,000 tonne capacity at
1.04 tonnes/m.\cite{ doe_yucca_2002}

Specific Temperature Change analysis by Radel, Wilson, et al. find a maximum
thermal capcity of 1.09 tonnes/m for commercial SNF (at an ELF of $49
GWd_{th}/m$).\cite{radel_effect_2007} 

Elongation of cooling times has the potential to expand the capacity of the
repository. `Cooling time' refers to delaying complete loading of the
repository. Longer cooling times allow high heat, short lived isotopes to decay
to lower activity before they begin to heat the repository. Much of the benefit
to repository capacity comes from the advantage that the cooling time allows a
decrease in the space between emplacement drifts. Aged SNF has lower heat flux
and so, the drift spacing can be decreased from 81 to 70 meters. A study by
Man-Sung Yim and colleagues at North Carolina state found that for a
representative commercial SNF composition a cooling time of 75 years allows for
over 100 MTU SNF disposal without expanding the Yucca Mountain
footprint.\cite{li_examining_2007}

Similarly, age based fuel mixing also allows for decreases in drift spacing. In
aged based fuel mixing, aged (long cool time) SNF is loaded in a mixture with
young SNF. This age based fuel mixing has been shown to achieve a $48\%$
increase in the repository capacity as constrained by heat
load.\cite{nicholson_thermal_2007} This factor uses a fiducial default
footprint of $4.6 km^2$ used in the NRC TSPA.  The reported $48\%$ increase in
capacity results in total repository capacity of 103,600
tonnes.\cite{williams_contract_2001}

In addition to variable drift spacing, other modifications to repository layout
have had promising results in terms of heat-limited repository capacity. The
Electric Power Research Institute (EPRI) in their Room at the Mountain study
found that with redesign of the repository an increased capacity of at least
$400\%$ (295 tonnes once-through SNF) and up to $900\%$ (663 tonnes) could be
expected to be achieved. Proposed design changes include decreased spacing
between drifts, a larger areal footprint, vertical expansion into second and
third levels of repository space, and hybrid solutions involving combinations
of these ideas. In particular, EPRI suggests either an expansion of the
footprint with redesign of the current Upper Block line load design plan or a
multi-level plan that repeats the footprint and line load design of the current
plan.\cite{kessler_room_2006}

\begin{table}[h!]
  \centering
      \footnotesize{
      \begin{tabularx}{\textwidth}{|X|c|c|X|}
          \multicolumn{4}{c}{\textbf{Yucca Mountain Footprint Expansion Calculations}}\\
          \hline
          Author&Max. Capacity&Footprint&Details\\
          &$tonnes$&$km^2$&\\
          \hline
          &&&\\
          OCRWM&$70,000$&$4.65$&``statutory case''\\
          &$97,000$&$6$&``full inventory case''\\
          &$119,000$&$~7$&``additional case''\\
          \hline
          &&&\\
          Yim, M.S.&$75,187$&$4.6$&SRTA code\\
          &$76,493$&$4.6$&STI method\\
          &$95,970$&$4.6$&$63$m drift spacing\\
          &$82,110$&$4.6$&75 yrs. cooling\\
          \hline
          &&&\\
          Nicholson, M.&$103,600$&$4.6$&drift spacing\\
          \hline
          &&&\\
          EPRI&&&\\
          &$63,000$&$6.5$&Base Case CSNF\\
          option 1&$126,000$&$13$&expanded footprint\\
          option 2&$189,000$&$6.5$&multi-level design\\
          option 3&$189,000$&$6.5$&grouped drifts\\
          options 2+3&$252,000$&$6.5$&hybrid\\
          options 1+(2or3) &$378,000$&$13$&hybrid\\
          options 1+2+3 &$567,000$&$13$&hybrid\\
          \hline
        \end{tabularx}
        \caption[Yucca Mountain footprint expansion calculations.]{Various analyses based on heat 
        load limited repository designs have resulted in footprint expansion calculations of the 
        YMR.} 
        \label{tab:footprint}
        }
      \end{table}
 


\subsection{Specific Temperature Integral}

Line loading ($t/m$) and areal power density ($W/km^2$) are two common metrics
for describing the fullness of the repository. While these metrics are
informative for mass capacity and power capacity respectively, they fail to
reflect differences in thermal behavior due to varying SNF compositions.  A
closer look at the isotopics of the situtation has proven much more applicable
to thermal performance studies of the repository, and the preferred method in
the current literature relies on specific temperature integrals.


Specific Temperature Integrals model the thermal source as linear along the
repository drifts, similar to the line loading and areal power density metrics.
However, a temperate integral takes account of heat transfer behavior in the
rock, includes the effects of myriad SNF compositions, and gives the thermal
integration over time for any specific location within the rock.  Man-Sung Yim
calls this the Specific Temperature Increase method\cite{li_specific_2008}
though other researchers have other names for this method. Tracy Radel calls
her temperature metric at a point in the rock the Specific Temperature
Change.\cite{radel_repository_2007}

In a repository with linear drifts, the Heat flux from the drifts can be
expressed as the superposition of the linear heat flux contributions of all the
radionuclides in the waste. Each radionuclide contributes in proportion to its
decay heat generation and its weight fraction of the SNF. With information
about isotopic composition of the SNF, the Specific Temperature Increase can
determine the maximum thermal capacity of the repository in terms of tonnes/m.
The length based accounting in $\frac{t}{m}$ is converted to
$\frac{t}{Repository}$ by multiplication with the total emplacement tunnel
length of the repository.  In the case of Yucca Mountain, this was 67 km.

\section{Detailed Computational Models of Nuclide Transport}
\label{sec:detailed_nuclide}

%%%%%%%%%%%%%%%% Detailed Nuclide Transport %%%%%%%%%%%%%%%%%%%%%%%%

\subsection{Li Model\cite{li_methodology_2006}} As a function of time, water
enters the Engineered Barrier System and corrodes the waste packages.  These
fail and from the failed waste packages nuclides are released according to
advective transfer.  Further transportation through the near and far field rock
medium is modeled in two modes, one representing the Unsaturated Zone, and one
representing the Saturated Zone.

\paragraph{Waste package failure and nuclide release} are modeled with two TSPA
code modules called EBSFAIL and EPSREL. The waste package failure rate is
determined from EBSFAIL which incorporates waste form chemistry, humidity,
oxidation, etc and upon contact from water begins the degradation process. The
results of EBSFAIL become the input to EBSREL which models corresponding
nuclide release from those failed waste packages. Mass balance governing the
nuclide release rate in this model allows advective transfer to dominate and
takes the form:

\begin{equation}
\dot{m_i}=w_{li}(t)-w_{ci}{t}-m_i\lambda_i+m_{i-1}\lambda_{i-1}\nonumber
\end{equation}

In this expression, $w_{li}(t)$ is the rate $[mol/yr]$ of isotope \emph{i}
leached into the water.  It is a function of water flow rate, chemistry, and
isotope solubility. $m_i$ describes the mass of isotope \emph{i}, and
$\lambda_i$ $[s{-1}]$ describes its decay constant. Finally, $w_{ci}(t)$
describes the advective transfer rate $[mol/yr]$ of the isotope \emph{i}. This 
model defines $w_{ci}$ as:

\begin{equation}
  w_{ci}(t)=C_i(t)q_{out}(t) 
\end{equation}

where $q_out$ is the volumetric flow rate of the water $[m^3/yr]$, and 
$C_i$ is the concentration on isotope $i$ in the waste package volume 
$m_i/V_{wp}$ in $[mol/m^3]$. These assumptions fail to take into account any
differences in the varying solubilities of the isotopes, but are quite
sensitive to the concentration of an isotope \emph{i} in the waste package
volume.  
characteristics and is independent of or weakly dependent on concentration.  

\subsection{ANDRA Dossier 2005} The ANDRA Dossier 2005 studies provided
detailed nuclide transport calculations for both argillaceous clay and granite
formations.

\subsubsection{Clay/Shale} More complicated saturation and resaturation phenomena 
are neglected and it is assumed that the initial repository condition is fully
resaturated. This is a conservative simplification. Another conservative
assumption is one in which the evacuation disturbed zone does not heal. Rather,
it is modeled in its damaged state immediately after excavation. 

This model only tracks 15 nuclides of importance.  These are chosen to be those
with halflives over 1000 years and most toxicity or mobility.
\cite{andra_argile:_2005} The behavior of radionuclides in 
glassvitrification forms can be summarized by categorizing them into mobile,
intermediate, and well retained elements. 

%more details in inset 10-2, andra argile phenom. . 

%%%%%%%%%%%%%%%%% From Andra %%%%%%%%%%%% 
% mobile elements. They comprise the alkalis (sodium, lithium), molybdenum and 
% technetium. They are released congruently with the degradation of the vitreous 
% matrix. (Andra, 2005d). Of the chemical toxins, boron corresponds to this 
% category of elements.  

%intermediate elements are partially retained in the alteration layer when the 
%glass degrades. These elements are caesium, alkaline earth metals (calcium, 
%strontium, barium), silicon and aluminium (Andra, 2005d).


%Highly retained elements are insoluble at the basic pH imposed by the

%degradation of the glass. They include all the transition elements (iron, 
%nickel, zinc, zirconium, etc.) and lanthanides and actinides (Andra, 2005d ; 
%Fillet, 1987).  
%%%%%%%%%%%%%%%%%%%%%%%%%%%%%%%%%%%%%%%%%


Waste form dissolution and package release was assumed to be immediate for some
waste forms and corrosion rate based for those where appropriate data was
available. Vitrified waste package releases were either modeled with a simple
model or a two phase phenomenological model. 

%(pg198 for references to themath)

Transportation through the backfill is modeled as diffusive, with high
permeability.

The evacuation disturbed zone has both a fracture zone and a microfissure zone.

In the host formation, movement is dominated by diffusion, and advection is
modeled, but is negligible.

%advection dispersion equation is developed in the section entitled ``The 
%retention and transport of solutes.'' inset 10-11 gives solubility and sorption 
%values for important elements.


% from andra pg 202: 
% 
%The permeability values determined in situ or on undisturbed samples are 
%between 10-14 and 10-13 m/s. The structure of the medium suggests permeability 
%anisotropy and leads to cautiously adopting a value ten times higher for the 
%horizontal permeability.  The reference calculation therefore takes into 
%account a vertical permeability value of 5.10-14 m/s and a horizontal 
%permeability value of 5.10-13 m/s. The influence of temperature on the 
%permeability value is not included in calculations for the normal evolution 
%scenario, but it is taken into account in the altered evolution scenarios 
%because of its greater influence.  

Tools used in the ALLIANCES software platform include Castem, PorFlow and Trace
for hydraulic calculations and nuclide transport and Prediver and Colonbo for
waste package failure.

\subsection{SKB Model} SKB used ConnectFlow to model Forsmark particle
transport. It also used a model called MIKE SHE in conjunction with ConnectFlow
in order to model ``uses locally measured meteorological data to simulate
transient water flow in the saturated (groundwater) and unsaturated zones,
overland flow, and water losses due to evapotranspiration processes
(interception, evaporation and transpiration)''. CF contains an elaborate
description of flow within the rock, whereas the MS model provides more
detailed information on near-surface water flow paths and the associated
discharge paths.



\section{Detailed Computational Models of Heat Transport}
\label{sec:detailed_heat}

%%%%%%%%%%%%%%%% Detailed Heat Transport %%%%%%%%%%%%%%%%%%%%%%%%


\subsection{LLNL MathCAD Model}

This model, created at \gls{LLNL} for the \gls{UFD} campaign, is written in a 
combination of MathCAD and Excel. The model consists of two parts. The first is 
a MathCAD solution of the transient homogeneous conduction equation. The 
solution of this equation at the boundary of the EBS and the waste package is 
then treated as a boundary condition for the heterogeneous steady state solver, 
which advances the simulation and calculates values through the geometry. The 
process is then iterated with a one year resolution in order to arrive at a heat 
evolution over  the lifetime of the repository. This model seeks to inform heat
limited waste capacity calculations for each geology, for many waste package 
loading densities, and for many fuel cycle options.  


\subsection{Bauer SINDA\ G Model}

This model, created at Argonne National Lab by Ted Bauer uses a lumped 
capacitance solver, called SINDA.

The Bauer model is geometrically adjustable. 

The SINDA lumped capacitance solver solves a thermal circuit, for which 
conducting nodes may be of four types corresponding to the four modes of heat 
transfer. 

For heat flux rate through each conducting node, the SINDA lumped parameter 
model solves the Fourier equation,

\begin{align}
  \dot{q} &= \frac{1}{R}( T_i - T_j )\\
  \label{sindafourier}
  \intertext{where}
  R &= ~~\mbox{resistance}\nonumber\\
  T &= ~~\mbox{temperature}\nonumber\\
  \intertext{and}
  \dot{q} &= ~~\mbox{heat rate}.\nonumber
\end{align}

Lumps are connected by conduction, convection, radiation, and mass flow heat 
transfer links. Briefly, these are represented by

\begin{align*}
  R_{cond} &= \frac{L}{kA}\\
  R_{conv} &= \frac{1}{ h\cdot A}\\
  R_{mf}  &= \frac{1}{\dot{m}c_p}\\
  R_{rad}  &= \frac{1}{\sigma F_{ij}A\left[ T_i + T_A + T_j + T_A 
  \right]\left[(T_i+T_A)^2+(T_j+T_A)^2\right]}
  \intertext{where}
  &k= ~~\mbox{conductivity}[] \\
  &A= ~~\mbox{area} [m^2]\\
  &c_p=~~\mbox{specific heat capacity} []  \\\
  &h= ~~\mbox{heat transfer coefficient}[] \\
  &\dot{m}= ~~\mbox{mass transfer rate}[kg\cdot s^{-1}] \\
  &T_i= ~~\mbox{lump temperature} [^\circ C] \\
  &T_A= ~~\mbox{absolute temperature} [^\circ C] \\
  &F_{ij}= ~~\mbox{radiation interchange factor} [-] 
\end{align*}

\subsection{Finite Difference Codes}

An array of finite difference codes exist which have been used to make 
repository heat evaluations. See the table. 

\subsection{Finite Element Codes}

An array of finite element codes exist which have been used to make 
repository heat evaluations. See the table. 

\subsection{ANDRA/RED-IMPACT Codes?}

Codes used by ANDRA and other programs within the RED-IMPACT assessment may not  
fit in any of the categories above.

%%%%%%%%%%%%%%%%%%%%%%%%%%%%%%%%%%%%%%%%%%%% 
% This is where CapacityNotes wasinserted % 
%%%%%%%%%%%%%%%%%%%%%%%%%%%%%%%%%%%%%%%%%%%%




%%%%%%%%%%%%%%%%%%%%%%%%%%%%%%%%%%%%%% 
%%%%%% Andra Sub-Codes Table %%%%%%%%%
%%%%%%%%%%%%%%%%%%%%%%%%%%%%%%%%%%%%%% 

\begin{table}
\centering
\footnotesize{
\begin{tabular}{|l|l|}
  \multicolumn{2}{c}{\textbf{Detailed Nuclide Transport Models Used in the ANDRA analysis.}}\\
\hline
Models &                                          Codes\\
\hline
Hydrogeology and “particle tracking”    & Connectflow (NAMMU component, 3D modelling,\\
in continuous porous media              &  finite elements).\\
                                        &  Geoan (3D modelling, finite differences).\\
                                        &  Porflow (3D modelling, finite differences).\\
Hydrogeology and “particle tracking”    &  Connectflow (NAMMU component, 3D modelling,\\
in discrete fracture networks.          &  finite elements).\\
                                        &  FracMan (generation of discrete fracture networks) and\\
                                        &  MAFIC (hydraulic resolution of the networks, 3D,\\
                                        &  finite elements).\\
Transport in continuous porous media.   &  PROPER (COMP-23 component, modelling in\\
                                        &  segments of the engineered barrier, finite differences).\\
                                        &  Goldsim (volume modelling of engineered barriers).\\
                                        &  Porflow.\\
Transport in discrete fracture networks.&  PROPER (FARF-31 component, 1D modelling 1D\\
                                        &  stream tube concept).\\
                                        &  PathPipe (conversion of networks of tubes for transport)\\
                                        &  and Goldsim (modelling in networks of 1D pipes).\\
\hline
\end{tabular}
\caption[Particle Transport Codes Used in ANDRA Assessment]{Similar to the Total System Performance Assessment, ANDRA's analyses are a coupled mass of many codes. Table reprouced from Argile Dossier 2005 \cite{andra_argile:_2005}}
\label{tab:andra}
}
\end{table}


%%%%%%%%%%%%%%%%%%%%%%%%%%%%%%%%%%%%%%



%\section{Waste Form Models}


%\subsection{TAD Canisters} The canisters proposed for transportation, aging and 
%disposal are called TAD canisters.  They are two concentric cylinders of steel 
%and alloy22 inside and out respectively. 


%\subsection{Borosilicate Glass} Current borosilicate glass: Includes processing 
%chemicals from original separations, with U/Pu removed, but minor actinides and 
%Cs/Sr remaining Potential borosilicate glass: No minor actinides and/or no 
%Cs/Sr; Mo may be removed to increase glass loading of radionuclides; it has 
%alower volumetric heat rate


%\subsection{Glass Ceramic} Glass Ceramic:  This is glass-bonded sodalite from 
%Echem processing of EBR-II, and from potential future Echem processing of oxide 
%fuels o Metal Alloy: This includes subcategories


%\subsection{Metal Alloy} Metal alloy from Echem: Includes cladding as well as 
%noble metals that did not dissolve in the Echem dissolution Metal alloy from 
%aqueous reprocessing:  Includes undissolved solids and transition metal fission 
%products


%\subsection{Advanced Ceramic} Advanced Ceramic: An advanced waste form that 
%includes iodine volatilized during chopping, which is then gettered during 
%head-end processing of used fuels


%\subsection{Separated Streams} Other:  Examples include radionuclides removed 
%from other waste forms (e.g., Cs/Sr, I, C), as well as new waste forms such as 
%a salt waste form

%\subsection{Classes A, B, and C waste} Lower Than High Level Waste (LTHLW): 
%Includes Classes A, B, and C

%\subsection{GTCC LTHLW} % Greater Than Class C (GTCC)




