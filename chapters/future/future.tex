\chapter{Summary and Future Work}\label{ch:future}

\section{Summary}

\subsection{Motivation}

The need for this work has been shown by a summary of the current state of the 
art of fuel cycle simulator repository capabilities. The literature review 
concluded that most fuel cycle simulators lack repository analysis beyond basic 
mass tracking. The integrated radionuclide transport and thermal analysis to be 
pursued in this work will provide a currently unavailable tool for disposal 
system analysis. An immediate need for such a tool has been expressed in the 
\gls{UFD} campaign roadmap for this year in which an interface with the \gls{SA} 
campaign was noted as a primary goal. 

\subsection{Current Work}

Development of the \Cyclus fuel cycle simulator has generated a tool with which 
the systems analysis aspects of this work will be conducted. Development of the  
fundamental repository module concept appropriate for modular integration with 
\Cyclus has laid the foundation for a software effort which will deliver a 
library of disposal system component models capable of analyzing current 
disposal concepts of interest both domestically and internationally.

\section{Future Work}

\subsection{Base Case Development}

% Base Case Milestone

\subsubsection{Base Case Concept}

  % Concept

    % uniform unfractured permeable porous medium

      % reducing geochemistry
      The base case concept will model a generic, isotropic, permeable porous 
      geological medium with reducing geochemistry. 
  
    % WF : glass 

      % alteration

      % temperature limit

    % WF : uox 

      % cladding limit

      % corrosion

      Two canonical waste form components will be developed. One will be modeled with 
      a nuclide release model dominated by alteration-based dissolution.  Such
      a waste form component module will be appropriate for a borosilicate glass waste form.  The 
      other waste form  component module will  be modeled with a corrosion based  
      dissolution model. This will be appropriate for a 

    % WP : fractional

    % Buffer : PurelyDiffusive Transport 

    % Heat limits at buffer and rock boundary 

\subsubsection{Base Case Abstraction Analysis}

  % Regression Analysis

\subsubsection{Base Case Code Development}

  % Code Development

\subsubsection{Base Case Testing, Verification, and Validation}<++>

  % Testing

% Geology Extension Milestone

  % Concepts

    % Fracturation (granite)

    % Fast Pathways (borehole and salt)

    % Coalescent behavior (salt)

  % Regression Analysis

  % Code Development

  % Testing


\subsection{Sensitivity Analysis}

A sensitivity analysis underway to arrive at simplified dominant physics 
relationships between the various input parameters of the simulation. 

\subsubsection{Nuclide Transport}

These are the independent parameters I'm interested in.

A TABLE, perhaps. Including parametric domain.

This is how I'll vary them.

Show that these provide a largely complete set of input variables to define 
nuclide transport. Discuss variables that are being neglected for simplicity, 
perhaps to be added later. 

\subsubsection{Heat Evolution}

These are the independent parameters I'm interested in.

A TABLE, perhaps. Including parametric domaie.

This is how I'll vary them.

Show that these provide a largely complete set of input variables to define heat 
evolution. Discuss variables that are being neglected for simplicity, perhaps to 
be added later. 

\subsection{Mathematical Model Abstraction}

Using the results of these sensitivity analyses, regression analysis will be 
performed to develop simplified parametric dependencies for all independent 
variables for each model. 

\subsection{Computational Model Development}

The development of the computational model will begin with simplistic models of  
the fundamental components of the detailing the phyical phenomena at work vs.  
the conceptual and mathematical models available, listed in order of complexity.


\subsubsection{Base Case}

Clay, bentonite backfill, no evacuation disturbed zone, some handfull of waste 
packages and waste forms.

%        File: cat_table.tex
%     Created: Tue Jul 19 11:00 AM 2011 C
% Last Change: Tue Jul 19 11:00 AM 2011 C
%
\begin{table}
  \centering
  \footnotesize{
  \begin{tabular}{|l|c|c|c|c|c|c|c|}
    \multicolumn{8}{c}{\textbf{Categorization of Phenomena}}\\
    \hline
     Phenomenon&Simplest&&&&&&Hardest\\
    \hline
     WF dissolution&instant&fractional&f(t)&f(H20)&f(T)&f(T,H20)&f(T,H20,etc.)\\
     WP dissolution&instant&fractional&f(t)&f(H20)&f(T)&f(T,H20)&\\
     WF release&instant&fractional&diffusive&advective&diff+adv&congruent&solubility limited\\
     WP release&instant&fractional&diffusive&advective&diff+adv&congruent&solubility limited\\
     Buffer failure&instant&fractional&f(t)&f(H20)&f(T)&f(T,H20)&f(T,H20,etc.)\\
     Buffer release &instant&fractional&diffusive&advective&diff+adv&congruent&solubility limited\\
     FF transport &diffusive&+fractures&+advective&congruent&+sorption&+colloids&solubility limited\\
     WF Heat&indexed&decay&&&&&\\
     WP Heat&conductive&+conv&+rad&+mass&2d&finite diff&finite element\\
     Buffer Heat&conductive&+conv&+rad&+mass&2d&finite diff&finite element\\
     FF Heat&conductive&+conv&+rad&+mass&2d&finite diff&finite element\\
    \hline
  \end{tabular}
  \caption[Categorization of Phenomena]{This table is a preliminary sketch of 
  the various categories of phenomena which will occur in the components of the  
  repository model.}
  \label{tab:cat}
  }
\end{table}





\paragraph{TAD Canisters} The canisters proposed for transportation, aging and 
disposal are called TAD canisters.  They are two concentric cylinders of steel 
and alloy22 inside and out respectively. 


\paragraph{Borosilicate Glass} Current borosilicate glass: Includes processing 
chemicals from original separations, with U/Pu removed, but minor actinides and 
Cs/Sr remaining Potential borosilicate glass: No minor actinides and/or no 
Cs/Sr; Mo may be removed to increase glass loading of radionuclides; it has 
alower volumetric heat rate


\paragraph{Glass Ceramic} Glass Ceramic:  This is glass-bonded sodalite from 
Echem processing of EBR-II, and from potential future Echem processing of oxide 
fuels o Metal Alloy: This includes subcategories


\paragraph{Metal Alloy} Metal alloy from Echem: Includes cladding as well as 
noble metals that did not dissolve in the Echem dissolution Metal alloy from 
aqueous reprocessing:  Includes undissolved solids and transition metal fission 
products


\paragraph{Advanced Ceramic} Advanced Ceramic: An advanced waste form that 
includes iodine volatilized during chopping, which is then gettered during 
head-end processing of used fuels


\paragraph{Separated Streams} Other:  Examples include radionuclides removed 
from other waste forms (e.g., Cs/Sr, I, C), as well as new waste forms such as a 
salt waste form

\paragraph{Classes A, B, and C waste} Lower Than High Level Waste (LTHLW): 
Includes Classes A, B, and C

\paragraph{GTCC LTHLW}  Greater Than Class C (GTCC)


\subsubsection{Demonstration}

Show that the complete model behaves in agreement with the more detailed model 
on which it was based. Else, iterate through sensitivity analyses, model 
abstraction, and computational development until the model is validated. 



\subsubsection{Extension}

This disposal system will at this point be extended to include a fleet of 
predefined model implementations to represent some canonical waste forms, 
packages, buffers, and clay types.  

\subsubsection{Fuel Cycle Analysis}

Show some various fuel cycles have different repository needs and metrics.  
Specifically, compare a closed fuel cycle, an open one, and at least one 
modified fuel cycle. 




% UFD developing metrics for the fct program option screening these and perhaps 
% other metrics will be included. 

