\chapter{Introduction}

The scope of this work includes implementation of a comprehensive 
software library of medium fidelity models to represent the thermal 
behavior and long-term disposal system performance of different 
disposal system concepts in different geologic media for deployment in 
a modular systems analysis platform.  This work will model repository 
behavior as a function of arbitrary spent fuel composition as well as 
modularly incorporate dominant physics disposal system models into a 
computational fuel cycle analysis platform.

\section{Motivation} 

The development of sustainable nuclear fuel cycles is a key challenge 
as the use of nuclear power expands internationally. While fuel cycle 
performance may be measured with respect to a variety of metrics, 
waste management metrics are of particular importance to the goal of 
sustainability. Since disposal options are heavily dependent on 
upstream fuel cycle decisions, a relevant analysis of potential 
geological disposal and waste package solutions therefore requires a 
system level approach. A simulation tool with top-level capability 
which will allow modular substitution of various fuel cycle facility, 
repository, and waste package models is needed. The development of 
such modular waste package and repository models will assist in 
informing current technology choices, identifying important parameters 
contributing to key waste disposal metrics, and highlighting the most 
promising waste disposal combinations. Specifically, such models will 
support efforts underway in the development of computational tools to 
quantify these metrics and understand the merits of different fuel 
cycle alternatives. 

System level fuel cycle simulation tools must facilitate efficient 
simulation of a wide range of fuel cycle alternatives as well as 
sensitivity and uncertainty analyses. Efficiency is achieved with 
models at a level of detail which successfully captures significant 
aspects of the underlying physics while acheiving a calculation speed 
in accordance with use cases requiring repeated simulations. Often 
termed abstraction, the process of simplifying while maintaining the 
salient features of the underlying physics is the method by which used 
fuel disposal system models are developed in this work. 

\subsection{Future Fuel Cycle Options}

Domestically, nuclear power expansion is motivated by the research, 
development, and demonstration roadmap being pursued by the United 
States Department of Energy Office of Nuclear Energy (DOE-NE) which 
seeks to ensure that nuclear energy remain a viable domestic energy 
option.  \cite{DOE-NE_roadmap} 

As the DOE-NE seeks to develop technologies and strategies to support 
a sustainable future for nuclear energy, various fuel cycle strategies 
and corresponding disposal system options are being considered. 
Specifically, the domestic fuel cycle option space under current 
consideration is described in terms of three distinct fuel cycle 
categories with the monikers Once Through, Full Recycle, and Modified 
Open, each category presenting unique disposal system siting and 
design challenges. Systems analyses for evaluating these options must 
be undertaken in order to inform a national decision to deploy a fuel 
cycle system by 2050. \cite{DOE-NE_roadmap} 

A Once-Through Cycle represents the continuation of the business as 
usual case in the United States,  neglecting reprocessing, and 
presenting the challenges associated with high volumes of minimally 
treated spent fuel streams.  In a business as usual scenario, 
conventional power reactors comprise the majority of nuclear energy 
production and fuel takes a single pass through a reactor before it is 
classified as waste and disposed of. In the open cycle, no 
reprocessing is pursued, but research and development of advanced 
fuels seek to reduce waste volumes, but calculations from the Electric 
Power Research Institute indicate that such a cycle will generate a 
volume of spent fuel that will necessitate the siting of two or more 
federal  geological repositories to accomodate spent fuel.  
\cite{Room_at_the_mountain} 

A Full Recycle option, on the other hand, requires theresearch, 
development, and deployment of partititioning, transmutation, and 
advanced reactor technology for the reprocessing of used nuclear fuel.  
In this scheme, conventional once-through reactors will be phased out 
in favor of fast reactor and Gen-IV reactor technologies with 
transmutation capacity and greater fuel efficiency. All fuel in the 
Fully Recycle strategy will be reprocessed and cycled through a 
reactor numerous times and may undergo partitioning and waste 
treatment before ultimate disposal in a repository. Such a repository 
under the Full Recycle scenario must adaptively support highly 
variable waste stream composition, as well as myriad waste forms and 
packaging associated with isolation of differing waste streams.

Finally, the Modified Open Cycle category of options includes a 
variety of fuel cycle options that fall between fully closed and fully 
open. Advanced fuel cycles such as deep burn and small modular 
reactors will be considered within the Modified Open set of fuel cycle 
options as will partial recycle options. Partitioning and reprocessing 
strategies, however, will be limited to simplified chemical 
separations and volitilization under this scheme. Spent fuel volumes 
and composition will vary dramatically among various possibilities 
within this scheme.  

Clearly, the range of waste streams resulting from potential fuel 
cycles present an array of corresponding waste disposition, packaging, 
and engineered barrier system options. A comprehensive analysis of the 
disposal system, dominant physics models must therefore be developed 
for these subcomponents.  Differing spent fuel composition, 
partitioning, transmutation, and chemical processing decisions 
upstream in the fuel cycle demand differing performance and loading 
requirements of waste forms and packaging. The capability to model 
thermal and nuclide transport phenomena through, for example, 
vitrified glass as well as ceramic waste forms with with various 
loadings for arbitrary isotopic compositions is therefore required.  

\subsection{Future Waste Disposal System Options}

In addition to the reconsideration of the domestic fuel cycle policy, 
the uncertain future of the Yucca Mountain repository has driven the 
expansion of the option space of potential repository geologies to 
include, at the very least, granite, clay, shale, salt, and deep 
borehole concepts. The
physical, hydrogeologic, and geochemical mechanisms which dictate 
nuclide and heat transport vary between these geological systems. 
Therefore, in support of the simulation effort, models must be 
developed which capture the salient physics of these geological 
options and quantify associated disposal metrics and benefits.  
Furthermore, modular linkage between subcomponent process modules and 
the repository envrionmental model must acheive a cohesively 
integrated disposal system model.  

\subsection{Heat Constraints}

Heat limits within a used nuclear fuel disposal system are waste form, 
package, and geology dependent. Heat generation from the waste form 
and transport through the engineered barrier system and host 
environment constrains fuel loading in waste forms and packages as 
well as placing requirements on the size, design, and loading strategy 
in a potential geological repository.

Heat load limits of various waste forms have their design basis in the 
temperature dependence of isolation integrity of the waste form. The 
waste form dissolution behavior under heat generation from the 
contained isotopics constrains loading density within the waste form 
in an isotope specific way.  

Heat load limits of various engineered barrier systems similarly have 
a design basis in the temperature dependent dissolution rate of the 
materials from whence they are constructed.  

Heat load limits of the geologies depend less on rock matrix 
degradation than on hydrogeology within the rock under heat evolution. 
The two heat load constraints which primarily determined the 
heat-based SNF capacity limit in the Yucca Mountain Repository design, 
for example, are fairly specific to unsaturated tuff. Thermal limits 
in that design are intended to passively steward the repository's 
integrity against radionuclide release for the upcoming 10,000 years.

The first constraint intends to prevent repository flooding and 
subsequent contaminated water flow through the repository. It requires 
that the minimum temperature in the granite tuff between drifts be no 
more than the boiling temperature of water which, at the altitude in 
question, is $96^{\circ}C$. This is effectively a limit on the 
temperature halfway between adjacent drifts, where the temperature 
will be at a minimum.

The second constraint intends to prevent high rock temperatures that 
induce fractures and would increase leach rates. It states that no 
part of the rock reach a temperature above $200^{\circ}C$, and is 
effectively a limit on the temperature at the drift wall, where the 
rock temperature is a maximum.  

Analgous constraints for a broader set of possible geological 
environments will depend on heat transport parameters of the matrix, 
hydrogeologic state, and repository drift spacing, waste package 
spacing, and repository footprint among other parameters. 

\subsection{Radiotoxicity and Source Term Constraints}

The exposure limit set by the NRC is based on a `reasonably exposed 
individual.' That is to say, the limiting case is a person who lives, 
grows food, drinks water and breathes air 18 km downstream from the 
repository. The Yucca Mountain Repository legistlative regulations 
limit total dose from the repository to 15 mrem/yr, and limit dose 
from drinking water to 4 mrem/yr.  Predictions of that dose rate 
depend on an enormous variety of factors. The primary pathway of 
radionuclides from an accidental release will be from cracking aged 
canisters. Transport of the radionuclides to the water table requires 
that the leakages come in contact with water and travel through the 
rock the water table. This results in contamination of drinking water 
downstream.  

\subsection{Radiotoxicity and Source Term Metrics}

Source term is a measure of the quantity of a radionuclide released 
into the
environment and radiotoxicity is a measure of the hazardous effect of 
that
particular radionuclide upon human ingestion. In particular, 
radiotoxicity is
measured in terms of the volume of water dilution required to make it 
safe to
ingest. Studies of source term and radiotoxicity therefore make 
probabilistic
assessments of radionuclide release, transport, and human exposure. 
The
probabilistic nature of these assessments mean a direct dependence of 
source
term on repository capacity can be difficult to arrive at. In order to 
give
informative values for the risk associated with transport of 
particular
radionuclides, for example, studies make hundred thousand year 
predictions
about waste form degredation, water flow, etc.  

A generalized metric of probablistic risk is fairly difficult to 
arrive at. The
Peak Environmental Impact metric from Berkeley 
\cite{bouvier_comparison_2007},
for example, is a complicated function of spent fuel composition, 
waste
conditioning, vitrification method, and radionuclide transport through 
the
repository walls and rock. Also, it makes the assumption that the 
waste
canisters have been breached at $t=0$. Furthermore, reported in $m^3$, 
PEI is a
measure of radiotoxicity in the environment in the event of total 
breach. While
informative, this model on its own does not completely determine a 
source-term
limited maximum repository capacity. Additional waste package failure 
and
exposed individual radiotoxicity constraints must be incorporated into 
it.


\section{Methodology} 

In this work, concise dominant physics models suitable for system 
level fuel
cycle codes will be developed from comparison of analytical models 
with more
detailed repository modeling efforts. The ultimate objective of this 
effort is
to develop a software library capable of assessing a wide range of 
combinations
of fuel cycle alternatives, potential waste forms, repository design 
concepts,
and geological media.  

Categorization and characterization of phyisical mechanisms by which 
nuclide
and thermal transport take place within the materials and media under
consideration will first be undertaken. In this way, the domain of
applicability for which subprocesses may be generalized will be 
assessed. In so
doing, a preliminary set of combinations of fuel cycles, waste forms,
repository designs, and geologies can be chosen which covers a 
foundational
subspace of the parametric domain. Once complete, extended models and 
model
variants will be developed to more comprehensively cover the option 
space.  

In general, such concise models are a combination of two components:
semi-analytic mathematical models that represent a simplified 
description of
the most important physical phenomena, and semi-empirical models that 
reproduce
the results of detailed models.  By shifting the emphasis between the
complexity of the analytic models and regression against numerical 
experiments,
variations can be limited between two models for the same system.  
Different
approaches will be compared in this work, with final modeling choices 
balancing
the accuracy and efficiency of the possible implementations.  


\section{Outline}

The following chapter will present a literature review which organizes 
and
reports upon previous relevant work. It will focus upon current 
analytical and
computational modeling of nuclide and heat transport through glass and 
ceramic
waste forms, engineered barrier systems, and geologies of interest. It 
will
also address previous efforts in generic geology repository modeling 
and the
state of the art of repository modeling and integration within current 
systems
analysis tools. 

Chapter \ref{ch:categorization} will categorize and characterize 
detailed
computational models of nuclide and heat transport available for 
regression
analysis. Specificially, detailed codes in current use are categorized
according to the physics which they model, the disposal system 
components with
which they are concerned, and the level of detail and computational 
methodology
with which they capture physical phenomena. 

Chapter \ref{ch:repository} will detail the analytical and regression 
analysis
undertaken to acheive a generic repository model for the chosen base 
repository
type. A concise, dominant physics geological repository model of the 
base case
disposal environment will be developed. Informed by semi-analytic 
mathematical
models representing important physical phenomena, existing detailed
computational efforts characterizing these repository environments 
will be
appropriately simplified to create concise computational models. This
abstraction will capture fundamental physics of thermal, 
hydrogeologic, and
radionuclide transport phenomena while remaining sufficiently detailed 
to
illuminate behavioral differences between each of the geologic systems 
under
consideration.  

These models will focus on the hydrogeology and thermal physics 
dominating
nuclide transport and heat response in candidate geologies as a 
function of
isotope release and heat generation over long time scales from waste 
packages.
Disposal site water chemistry (redox state) and dominant transport 
mechanism
(advection or diffusion) will provide primary differentiation between 
the
different geologic media under consideration. In addition, the concise 
models
will be capable of roughly adjusting release pathways according to the
characteristics of the natural system (both the host geology and the 
site in
general) and the engineered system (such as package loading 
arrangements,
tunnel spacing, and engineered barriers).

The abstraction process will employ the comparison of semi-analytic 
thermal and
hydrogeologic models and analytic regression of rich code results as 
well as
existing empirical geologic data.  For example, analytic models and
semi-empirical models are available (i.e. specific temperature 
integrals \cite{
} or specific temperature change \cite{radel_wilson_2004}) which 
approximate
the thermal response from heat generation in the waste packages as 
linear along
the repository drifts, and arrive at thermal evolution over time at 
any
location in the rock as well as line loading and areal power density 
metrics.
Such analytic models will first be assessed to determine likely 
parameters upon
which thermal response will rely (e.g. tunnel spacing, nuclide 
inventories,
etc.).  At this point, a regression analysis concerning those 
parameters will
be undertaken with available detailed models (e.g. 3D finite element 
codes and
full performance assessment models) to further characterize the 
parametric
dependence of thermal loading in a specific geology. Finally, the 
thermal
behavior of a repository model so developed will depend on empirical 
data (e.g.
heat transfer coefficients, water presence etc.). Determination of 
appropriate
values to make available within the dominant physics model will rely 
on
existing empirical data concerning the specific geologic environment 
being
modeled (i.e. clay, salt, shale, granite). A similar process will be 
followed
for nuclide transport models.

Verification and validation of these repository models will be 
conducted
through iterative benchmarking against more detailed repository 
models.

Chapter \ref{ch:ebs} will adapt existing models and data to the 
development of
concise dominant physics waste form, waste package, and other 
engineered
barriers (i.e., bentonite or cementitious materials) models 
appropriate for
treatment of key radionuclides within the waste streams.  
Material/barrier
degradation, radionuclide release, and radionuclide transport, and 
thermal
processes and effects will be included, as necessary, in the concise
representations that will be developed for subsequent use in the 
system-level
architecture. A range of waste forms, waste package materials, and 
other
engineered barrier materials (buffer, backfill) under consideration by 
the
DOE-NE FCT program (SWF and UFD campaigns) will be evaluated. The 
concise
dominant physics models will include appropriate load limiting factors 
such as
waste characterization, classification, stabilizing medium 
limitations, and
waste form heat loading behavior.

The abstraction process will employ the comparison of semi-analytic 
models and
analytic regression of rich code results as well as existing empirical 
material
data. For instance, in the case of nuclide release for waste packages, 
analytic
models of nuclide release (e.g. congruent or solubility limited) will 
first be
assessed to determine likely parameters upon which nuclide release 
will rely
(e.g. nuclide concentration, water flow rates, etc.) 
\cite{ahn_congruent}. At
this point, a regression analysis concerning those parameters will be
undertaken with available detailed models to further characterize the
parametric dependence of nuclide release from specific waste packages.
Finally, the nuclide release model so developed will depend on 
empirical data
(e.g. the waste package dissolution rate). Determination of 
appropriate values
to make available within the dominant physics model will rely on 
existing
empirical data concerning the specific package materials being modeled 
(i.e.
extrapolations of known material failure rates).  

Feedbacks from the geologic media response model on the response of 
the
engineered system, including waste forms, will have to be considered 
carefully.
In particular, given the important role of temperature in the system, 
thermal
coupling between the models for the engineered system and the geologic 
system
will be important. Thermal dependence of nuclide release and transport 
as well
as package degradation will necessarily be analyzed to determine the 
magnitude
of coupling effects in the system.

Chapter \ref{ch:extension} will discuss the future work necessary to 
extend
developed models to comprehensively cover the potential disposal 
system option
space. The path forward for extension of the geological base case 
model to
cover all five geologic concepts of interest (clay, granite, salt, 
shale, and
deep boreholes) will be discussed. Similarly, gaps in waste form and 
engineered
barrier system models and data will be addressed and a plan for data 
and model
coverage for that options space will be described.

Chapter \ref{ch:conclusion} will summarize the conclusions reached 
concerning
the appropriate analytical and detailed models to utilize in the 
process of
abstraction for nuclide and heat transport through various components 
of the
disposal system. Categorization of models and determination of the 
option space
parametric domain coverage will also be summarized. Finally, remaining 
future
work and expected contributions to the field will be summarized. 
