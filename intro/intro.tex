\chapter{Introduction}
The scope of this work includes implementation of a comprehensive software library of
medium fidelity models to represent the thermal behavior and long-term disposal system
performance of different disposal system concepts in different geologic media for deployment in a 
modular systems analysis platform. This work will model repository behavior as a function of 
arbitrary spent fuel composition as well as modularly incorporate dominant physics process models 
into a computational fuel cycle analysis platform.

\section{Motivation}



\section{Methodology}
This work will follow a phased approach leading from the the development of concise dominant physics 
models suitable for system level fuel cycle codes will arise from comparison of analytical models 
with more detailed repository modeling efforts. The ultimate objective of this effort is to
develop a software library capable of assessing a wide range of combinations of fuel cycle
alternatives, potential waste forms, repository design concepts, and geological media. 
%However, the overall approach will be to follow this phased path by first selecting a limited set 
%of
%combinations. Once complete, additional variants will be developed, again through the phases
%discussed below. This approach will allow the model development process to take advantages
%of lessons learned in the initial development of the process and tool architecture, allowing for
%improved efficiency in subsequent model development.

In general, such concise models are a combination of two components: semi-analytic
mathematical models that represent a simplified description of the most important physical 
phenomena, and semi-empirical models that reproduce the results of detailed models.
By shifting the emphasis between the complexity of the analytic models and regression against 
numerical experiments, variations can be limited
between two models for the same system.  Different approaches will be compared in this
work, with final modeling choices balancing the accuracy and efficiency of the possible
implementations.

\section{Outline}


